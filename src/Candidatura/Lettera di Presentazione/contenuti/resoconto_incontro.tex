\section{Resoconto dell'incontro}

In data 21 ottobre 2024 si è svolto un incontro con due rappresentanti dall'azienda proponente.
L'incontro è avvenuto tramite piattaforma Zoom, ed è stato utile per rispondere ai seguenti dubbi emersi nel gruppo:

\begingroup
\renewcommand{\ni}{\noindent}

  \vspace{0.5cm}

  \begin{tabular}{>{\justifying\arraybackslash}p{0.5\textwidth} >{\justifying\arraybackslash}p{0.6\textwidth}}
      \multicolumn{1}{c}{\textbf{Domande}} & \multicolumn{1}{c}{\textbf{Risposte}} \\ \\
      
      \ni 1. Come deve essere strutturata la parte web? (nello specifico: deve essere disponibile anche una versione mobile-friendly? il sito deve essere 
      integrato in un sito vostro già esistente?l'interazione sarà solo testuale oppure deve permettere l'iintegrazione di caricamento di un file?)
      & \ni Frontend deve essere un’interfaccia in cui fare l’inserimento della domanda e vedere la risposta.
      Tutta la parte dell’autenticazione è oggetto di sviluppi futuri.
      La parte principale non è ne il frontend ne il backend, ma quella che comunica con le 3 piattaforme e gestisce la risposta. La parte web può essere 
      una cosa anche molto base.
      Non serve una versione mobile e non provedere il caricamento di file (può essere buono ma non è obbligatorio). \\ \\
      
      \ni 2. Puo’ essere un’idea usare un prompt di sistema che aggiunge informazioni alla domanda inviata? (personalizzazione della domanda tramite l'uso 
      di un prompt di sistema)
      & \ni Questa parte non è stata valutata più di tanto: per il momento è un dettagliare una funzionalità che potenzialmente potrebbe essere uno
      sviluppo ulteriore ma non quello principale, è qualcosa in più. Cosa che però dato che si tratta di un chatbot interno potrebbe essere hard-coded. \\ \\
      
      \ni 3. Volete sicurezza e oscuramento dei dati sensibili?
      & \ni È un discorso che va nei coni di visibilità, quello è interessante, ma riteniamo sia già complessa la parte di integrazione coi 3 sistemi:
      si apre un mondo di capire come fanno due utenti a vedere solo una parte di documenti non “allineati”.
      Il capitolato è già abbastanza complesso per prevedere certe cose \\ \\

      \ni 4. Parlando di persistenza del database, quante domande per ogni chat devono essere salvate all'interno del db? (Nello specifico, possiamo
      mantenere nello storico chat un numero limitato di interazioni?)
      & \ni Nel database man mano che vengono date domande e risposte devono essere salvate una e l’altra. In futuro è possibile implementabile un feedback.
      Non c'è la questione dell'utenza, c'è un'unica utenza in cui viene mostrato tutto. Possibilità: storico di sessione (prossima sessione tutto vuoto)
      oppure "ultime 20 o 50 risposte" o "scroll" (scrolli in alto e si visualizzano le interazioni passate). Non perdete tempo sul frontend, 
      meglio includere Slack e Telegram piuttosto. \\ \\
  \end{tabular}

  \begin{tabular}{>{\justifying\arraybackslash}p{0.5\textwidth} >{\justifying\arraybackslash}p{0.6\textwidth}}
      \ni 5. Ci potete fornire dei comportamenti tipo in modo da istruire il bot? (nello specifico, definiamo delle regole che il bot deve seguire)
      & \ni Le piattaforme sono incrociate tra loro. Su Confluence sono scritte le specifiche. Jira raccoglie i vari ticket. Su GitHub ci sono gli sviluppi 
      di codice (con il commit che referenzia il ticket di Jira). Il chatbot deve essere istruito a collegare queste piattaforme tra loro. \\ \\
      
      \ni 6. Ci potreste dare APIkey di confluence, git, jira? allo stesso modo APIkey di OpenAI?
      & \ni Per Confluence e Jira c'è la versione gratuita, fino a 10 utenti. Quindi possiamo procurarci un nostro spazio Confluence e Jira. 
      GitHub: possiamo usare anche una repo generica o una trovata online. Per OpenAI c’è una questione di token utilizzabili, ma ce li metteranno loro 
      a disposizione (volendo ci sono anche altre versioni open source). Noi avremo dei nostri spazi in cui ci verranno forniti i documenti con cui allenare
      il bot, certamente non avremo accesso a tutto il loro Confluence. Su Confluence noi saremo utenti guest. Jira: in caso, noi comunque non avremo accesso 
      a tutta la documentazione dei progetti, potremo avere una configurazione ad hoc, un ambiente tutto nostro, per non accedere ai loro dati. \\ \\

      \ni 7. Per mostrarvi la documentazione tecnica, il bug reporting, il codice sorgente ci potreste  aprire su confluence,su git e su jira  una parte 
      esclusiva per noi? (nello specifico, uso dei vostri sistemi per una comunicazione diretta)
      & \ni Non chiedono il tipo di report bug dove l'utente riporta un bug del bot. Per la documentazione tecnica, forse ci danno uno spazio su 
      Confluence e Jira, ci diranno alla riunione introduttiva che faranno con i due gruppi che si saranno aggiudicati il progetto. 
      Potenzialmente possono loro aprire la repo sui loro spazi e direttamente farci usare la sezione "Issues" di GitHub. Loro scrivono i bug, 
      noi li risolviamo. \\ \\
      
      \ni 8. Ci potreste fornire degli schemi di come avete organizzato la documentazione su confluence, il codice su git e quanto specifico è il vostro sistema di ticket su jira?
      & \ni Confluence è simile a documenti Markdown, è come un Drive. Del resto, ne si parla alla riunione introduttiva. I ticket su Jira hanno nome
      (5-6 parole), richiesta, descrizione e allegati. Buddybot non deve gestire allegati, deve solo leggere la descrizione del ticket. Nulla da dire di 
      particolare sulla struttura del loro spazio Jira, dunque dovremo allenare il bot ad accedere ad una normale board di Jira.
      In GitHub, hanno una repo di backend (TypeScript) e una di frontend (JavaScript), non c’è nulla di particolare sulla struttura. \\ \\
  \end{tabular}

  \begin{tabular}{>{\justifying\arraybackslash}p{0.5\textwidth} >{\justifying\arraybackslash}p{0.6\textwidth}}
      \ni 9. Volete per forza un'architettura in microservizi? (Nello specifico, si possono usare software monolitici per risparmiare risorse?)
      & \ni Piena libertà sulla metodologia. Riconoscono che è più rapido fare in modo alternativo, quindi ce lo accettano. \\ \\
    \end{tabular}



In seguito è stata mantenuta una corrispondenza tramite mail per chiarire il seguente dubbio: \\ \\
\begin{tabular}{>{\justifying\arraybackslash}p{0.5\textwidth} >{\justifying\arraybackslash}p{0.6\textwidth}}
  \multicolumn{1}{c}{\textbf{Domanda}} & \multicolumn{1}{c}{\textbf{Risposta}} \\ \\
  \ni Saremmo grati se poteste fornirci qualche indicazione strategica sul progetto
  & \ni Focalizzatevi su progettazione e sviluppo, poi anche il resto dovrà essere realizzato ma sicuramente le due fasi menzionate sono quelle chiave. \\ \\
\end{tabular}

\endgroup