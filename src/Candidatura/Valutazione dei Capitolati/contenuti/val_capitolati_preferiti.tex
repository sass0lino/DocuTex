\section{Valutazione dei Capitolati preferiti}


\subsection{C7: LLM: Assistente digitale}

\subsubsection{Descrizione}
Il Progetto prevede la realizzazione di un assistente virtuale 
che aziende il cui core business è dato dalla vendita di prodotti
possono mettere a disposizione dei propri clienti, 
al fine di facilitare la ricerca di informazioni sui prodotti disponibili 
e rispondere alle domande più frequenti.
\subsubsection{Dominio}
\paragraph{Dominio tecnologico:}
Non ci sono tecnologie obbligatorie, ma sono state consigliate:
\begin{itemize}
    \item \textit{\textbf{MySQL}}: per la gestione del database relazionale.
    \item \textit{\textbf{LLM}}: il proponente ha indicato un insieme di modelli open source, tra questi si evidenzia Italia by iGenius, modello con 9 miliardi di parametri addestrato con un dataset al 90\% italiano.
    \item \textit{\textbf{API Rest}}: per la comunicazione tra il modello LLM e l'applicativo di interazione con l'utente.
    \item \textit{\textbf{ODBC}}: Open Database Connectivity, standard per la comunicazione da e per il database.
    \item \textit{\textbf{.NET MAUI}}: framework per lo sviluppo di applicazioni cross-platform.
\end{itemize}
\paragraph{Dominio applicativo:}
Il progetto si inserisce nel contesto della digitalizzazione e dell’uso del Machine Learning 
per analizzare grandi quantità di dati aziendali e sviluppare sistemi di interazione uomo-macchina avanzati. 
Nello specifico è rivolto alle aziende del settore della vendita di alimenti, 
dove la conoscenza dettagliata dei prodotti è spesso affidata a specialisti. 
Mira a creare un assistente virtuale che aiuti i clienti a trovare informazioni sui prodotti disponibili 
e risponda alle domande frequenti, migliorando l’accessibilità delle informazioni 
e l’efficienza del processo di assistenza clienti, riducendo la dipendenza dagli specialisti umani 
e offrendo risposte rapide e accurate alle domande dei clienti.
\subsubsection{Criticità riscontrate}
Fatta eccezione per l’elevata complessità delle tecnologie necessarie per lo sviluppo, 
che richiederà un certo periodo di studio da parte di tutti i membri del gruppo, 
non abbiamo riscontrato importanti criticità. 
Tuttavia ci sono alcuni aspetti che ci hanno convinto a scegliere il capitolato C9 piuttosto di questo. 
I due capitolati trattano progetti molto simili, ma riteniamo che per quanto riguarda il capitolato C9 
l’azienda proponente offra un supporto più adeguato alle nostre esigenze, 
e le tecnologie suggerite hanno suscitato maggiore interesse da parte del gruppo.
\subsubsection{Conclusioni}
Il gruppo ha apprezzato molto il progetto presentato, 
riconoscendone il valore e la potenziale utilità. 
Tuttavia, abbiamo deciso di non procedere con questo progetto, 
non per una mancanza di qualità, ma perché un altro capitolato ha suscitato un maggiore interesse da parte nostra.

\subsection{C2: Vimar GENIALE}

\subsubsection{Descrizione}
Nel capitolato si propone la realizzazione di un chatBot che gli installatori 
possono interpellare per reperire  informazioni testuali e grafiche sui prodotti 
Vimar presenti all’interno del sito ufficiale.
\subsubsection{Dominio}
\paragraph{Dominio tecnologico:}
Per lo svolgimento del capitolato ci sono delle tecnologie obbligatorie da utilizzare:
\begin{itemize}
    \item \textit{\textbf{Docker}}: Permette lo sviluppo di istanze isolate su cui testare il prodotto in un ambiente controllato
    \item \textit{\textbf{Git}}: Sistema di versionamento
    \item \textit{\textbf{Approcio RAG (retrieval Augmented Generation)}}: Permette il recupero di informazioni e generazione di risposte in linguaggio naturale
\end{itemize}
Oltre alle tecnologie obbligatorie abbiamo anche delle tecnologie consigliate:
\begin{itemize}
    \item \textit{\textbf{Angular}}:  Per realizzare il front-end della web app
    \item \textit{\textbf{Python}}:  Per realizzare le API e la business-logic
    \item \textit{\textbf{Scrapy}}: Utile per raccogliere dati da diversi siti web
    \item \textit{\textbf{PostgreSQL}}: Per database relazionale in cui immagazzinare i dati 
\end{itemize}
\paragraph{Dominio applicativo:}
Questo capitolato si propone per aiutare e facilitare il lavoro degli installatori 
con un applicativo web dove potranno contattare e reperire informazioni molto utili 
per agevolare l’installazione degli apparati.
Lo scopo di questo progetto è quello di realizzare un applicativo web simile come 
funzionamento ad un chat bot attraverso il quale gli installatori potranno chiedere 
in linguaggio naturale ed ottenere informazioni molto importanti come ad esempio: 
dati tecnici, schema elettrico, istruzioni funzionamento ecc.
  
\subsubsection{Criticità riscontrate}
\begin{itemize}
    \item \textit{\textbf{Minore libertà tecnologica}}:  L’azienda è più rigida nelle tecnologie che richiede di utilizzare nel capitolato rispetto ad altri proponenti.
    \item \textit{\textbf{Poca reperibilità}}:  L’azienda si è resa meno disponibile rispetto agli altri proponenti poiché non ha accettato di tenere un incontro telematico, preferendo un documento condiviso. Il che ci ha dato un segnale negativo riguardo al supporto che potremmo ricevere durante la fase progettuale.
\end{itemize}
\subsubsection{Conclusioni}
Il capitolato C2 è un’ottima occasione per chi vuole imparare competenze avanzate 
e arricchire il proprio profilo. Tra i vantaggi principali ci sono l’opportunità 
di fare esperienza sia nel front-end che nel back-end (full-stack) e l’uso di 
tecnologie moderne e molto richieste. Tuttavia, viste anche le criticità sopra citate, 
abbiamo deciso di non procedere con questo progetto perchè un altro capitolato ha 
suscitato un maggiore interesse da parte nostra.