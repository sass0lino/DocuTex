% Configurazione
\documentclass{article}

\usepackage{titling} % Required for inserting the subtitle
\usepackage{graphicx} % Required for inserting images
\usepackage{tabularx} % Per l'ambiente tabularx (tabelle)
\usepackage{calc} % Sempre per le tabelle
\usepackage[hidelinks]{hyperref} % Per i collegamenti ipertestuali, ad esempio sulla table of contents
\usepackage[italian]{babel} % Per la lingua italiana nelle scritte automatiche
\usepackage{xcolor} % Per colorare il testo
\usepackage{colortbl} % Per colorare le celle delle tabelle
\usepackage{lipsum} % Per generare lorem ipsum
\usepackage[normalem]{ulem} % Per sottolineare il testo
\usepackage{array} % Per la visualizzazione fluttuante di array di domande e risposte
\usepackage{ragged2e} % Pacchetto necessario per \justifying che giustifica il testo di tabelle
\usepackage{tikz} % Per spostare un'immagine nel documento in modo facile e veloce

\newcommand{\ulhref}[2]{\href{#1}{\uline{#2}}} % Nuovo comando per sottolineare i link
\newcommand{\ulref}[1]{\uline{\ref{#1}}} % Nuovo comando per sottolineare i collegamenti a immagini
\setlength{\parindent}{0pt} % Rimuove il rientro automatico dei paragrafi

\graphicspath{ {immagini/} {../../../../shared/immagini/} }


% Variabili
\newcommand{\data}{22 Ottobre 2024}


% Struttura
\begin{document}

% Parte costante
\input{contenuti/intestazione}
{
\centering
\Huge\bfseries Verbale Riunione\par
\vspace{0.5cm}
\Large\bfseries \data\par
}
\newpage
\tableofcontents
\newpage

% Parte variabile
\begin{center}
\begin{tabular}{r|l}
    \textbf{Versione} & 1.0.0 \\
    \hline
    \textbf{Stato} & Approvato \\
    \hline
    \textbf{Redazione} & Michael Fantinato \\
                       & Riccardo Stefani \\
    \hline
    \textbf{Verifica} & Davide Verzotto \\
    \hline
    \textbf{Approvazione} & \emph{Tutto il gruppo} \\
    \hline
    \textbf{Proprietario} & \emph{SWEg Labs} \\
    \hline
    \textbf{Uso} & Esterno \\
    \hline
    \textbf{Destinatari} & Prof. Tullio Vardanega \\
                         & Prof. Riccardo Cardin \\
\end{tabular}
\end{center}

\newpage
% Intestazione
\fancyhead[L]{2 \hspace{0.2cm} Ordine del giorno} % Testo a sinistra


\section{Ordine del giorno}

\begin{itemize}
    \item Aggiornamento avanzamento lavori;
    \item Aggiornamento del \emph{consuntivo}\textsubscript{\textit{\textbf{G}}} per la \emph{sprint}\textsubscript{\textit{\textbf{G}}};
    \item Preparazione incontro con il professor Cardin del 18-12-24;
    \item Preparazione e organizzazzione incontro con \emph{AzzurroDigitale}\textsubscript{\textit{\textbf{G}}} del 19-12-24;
    \item Prima discussione sull'avanzamento minimo durante le vacanze di natale.
\end{itemize}

\newpage
% Intestazione
\fancyhead[L]{3 \hspace{0.2cm} Diario della riunione} % Testo a sinistra


\section{Diario della riunione}

\begin{itemize}
    \item Sono stati analizzati i documenti richiesti per la RTB e sono state formulate delle strategie e suddivisi i compiti per la loro stesura iniziale:
    \begin{itemize}
        \item E' stato analizzato il documento \emph{Norme di Progetto}\textsubscript{\textit{\textbf{G}}} e abbiamo stabilito che per il momento possiamo 
        scrivere alcune parti generali e descrittive, mentre in futuro faremo una revisione più approfondita delle norme che ci è richiesto stabilire.
        \item E' stato analizzato il documento \emph{Analisi dei Requisiti}\textsubscript{\textit{\textbf{G}}} e abbiamo stabilito che per il momento 
        possiamo scrivere solo la parte introduttiva, poichè infatti attendiamo di svolgere l'incontro esterno con 
        \emph{AzzurroDigitale}\textsubscript{\textit{\textbf{G}}} di martedì 12 novembre per avere una visione più chiara dei requisiti.
        \item E' stato analizzato il documento \emph{Glossario}\textsubscript{\textit{\textbf{G}}} e abbiamo stabilito che, mano a mano che scriviamo i 
        documenti, possiamo aggiungere i termini che riteniamo più importanti. Inoltre, abbiamo deciso di prepararci già una lista di termini che potrebbero 
        essere inclusi nel glossario, che inseriremo al momento dell'effettivo utilizzo.
        \item E' stato analizzato il documento \emph{Piano di Progetto}\textsubscript{\textit{\textbf{G}}} e abbiamo stabilito che per il momento possiamo 
        scrivere la parte introduttiva e iniziare l'\emph{analisi dei rischi}\textsubscript{\textit{\textbf{G}}} con quanto abbiamo riscontrato finora. 
        Inoltre, ci siamo resi conto dell'importanza di documentare le attività svolte in ogni periodo \emph{sprint}\textsubscript{\textit{\textbf{G}}}, 
        con ausilio di grafici e diagrammi, per cui abbiamo riconosciuto la necessità di esplorare la strumentazione legata a quell'ambito.
        La nostra prima sprint partirà dopo l'incontro esterno con \emph{AzzurroDigitale} di martedì 12 novembre, e per allora dovremo essere preparati a
        documentare il nostro operato di progetto come richiesto dallo stato dell'arte.
        \item E' stato analizzato il documento \emph{Piano di Qualifica}\textsubscript{\textit{\textbf{G}}} e abbiamo stabilito che per il momento possiamo 
        scrivere la parte introduttiva e poco altro, in quanto ancora ci manca la teoria delle metriche e degli obiettivi di qualità. Abbiamo tuttavia già 
        deciso le metriche riguardanti la documentazione, cioè gli obiettivi di correttezza linguistica e leggibilità, così da avere la possibilità di 
        verificare la documentazione che scriviamo in questo periodo.
    \end{itemize}
    \item Abbiamo discusso l'opzione \emph{Jira}\textsubscript{\textit{\textbf{G}}} per la gestione del \emph{backlog}\textsubscript{\textit{\textbf{G}}}, 
    poichè essa consente anche la creazione automatica di grafici e diagrammi di progetto. Abbiamo tuttavia valutato che non tutti i grafici e diagrammi 
    richiesti sono inclusi in \emph{Jira}, per cui diventa necessario utilizzare anche \emph{Fogli Google}\textsubscript{\textit{\textbf{G}}}, e allora 
    facendo un ragionamento di convenienza abbiamo valutato che è meglio utilizzare \emph{GitHub Projects}\textsubscript{\textit{\textbf{G}}} per la gestione
    del backlog, poichè esso è già incluso nel nostro ambiente di lavoro e ci permette di avere un'interfaccia unica per tutto il progetto, e 
    \emph{Fogli Google} per tutti i grafici e alcuni diagrammi. \emph{Jira}, infatti, rappresenta una via di mezzo tra i due strumenti che non consente però
    di approfondire adeguatamente i due ambiti, dunque non è conforme alle richieste del nostro progetto.
    \item Abbiamo discusso la possibilità di mantenere un repo unico per tutto il progetto, ma abbiamo tuttavia deciso che il nostro caso specifico non lo
    consente, poichè infatti abbiamo già un repo per la documentazione configurato e funzionante, munito di un'automazione che compila i documenti in PDF
    e li rende disponibili in una vista tramite \emph{GitHub Pages}\textsubscript{\textit{\textbf{G}}}, e integrare il codice in esso condurrebbe a 
    difficoltà nel gestire lo \emph{script}\textsubscript{\textit{\textbf{G}}} d'automazione.
    Abbiamo quindi deciso di mantenere due repo separati, uno per la documentazione e uno per il codice, e di utilizzare \emph{GitHub Projects} per la
    gestione del backlog di entrambi i repo.
    \item Ci siamo posti l'obiettivo di sistemare i documenti come sopra descritto entro la prossima riunione interna di lunedì 11 novembre, 
    in modo da avere tutto pronto per l'incontro esterno con \emph{AzzurroDigitale} del giorno dopo, e di poter cominciare con la prima sprint 
    di progetto dal giorno dopo ancora.
\end{itemize}

\newpage
\section{Todo}

Durante la riunione sono emersi i seguenti task da svolgere:

\vspace{0.5cm}

\begin{table}[htbp]
\begin{tabular}{|p{0.4\textwidth}|p{0.6\textwidth}|}
    \hline
    \rowcolor[gray]{0.9}
    \multicolumn{1}{|c|}{\textbf{Assegnatario}} & \multicolumn{1}{|c|}{\textbf{Task Todo}} \\
    \hline
    Righetto Filippo & Completare la Valutazione dei Capitolati inserendo nel documento LaTeX la valutazione di \emph{C2: Vimar GenIAle} \\
    \hline
    Verzotto Davide & Aggiungere l'ultimo contatto avuto con \emph{AzzurroDigitale} via mail alla Lettera di Presentazione \\
    \hline
    Bolognini Federica & Apportare le modifiche necessarie alla suddivisione dei ruoli e al costo totale dell'impegno nel Preventivo 
    dei costi e degli impegni orari, per far sì che l'impegno orario di ciascun membro equivalga a 92 ore. \\
    \hline
    \emph{[Tutto il gruppo, singolarmente]} & Indagare ed informarsi sulla documentazione richiesta per la Requirements and Technology Baseline, per 
    valutare quando partire a redigerla e quali sezioni scrivere per prime. \\
    \hline
    \emph{[Tutto il gruppo, singolarmente]} & Preparare delle domande per la lezione rovesciata di mercoledì \emph{6 Novembre 2024} a tema Documentazione. \\
    \hline
    Stefani Riccardo & Redigere verbale riunione \emph{30/10/24}. \\
    \hline
\end{tabular}
\end{table}


\end{document}
