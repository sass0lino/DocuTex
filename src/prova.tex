\documentclass[a4paper, 12pt]{article}

% ----- PREAMBOLO -----
% Impostazioni della lingua e della codifica
\usepackage[utf8]{inputenc}
\usepackage[italian]{babel}

% Pacchetti per la matematica
\usepackage{amsmath}
\usepackage{amsfonts}
\usepackage{amssymb}

% Pacchetto per inserire immagini
\usepackage{graphicx}

% Pacchetto per il testo segnaposto (Lorem Ipsum)
\usepackage{lipsum}

% Impostazioni della pagina
\usepackage[a4paper, top=3cm, bottom=3cm, left=2.5cm, right=2.5cm, marginparwidth=1.75cm]{geometry}

% Informazioni sul documento
\title{Documento di Prova in \LaTeX}
\author{Gemini}
\date{\today}


% ----- INIZIO DEL DOCUMENTO -----
\begin{document}

\maketitle

\begin{abstract}
    Questo documento serve come esempio pratico per dimostrare alcune delle funzionalità di base e intermedie del sistema di composizione tipografica \LaTeX. Verranno illustrati elementi come la strutturazione del testo, l'inserimento di formule matematiche complesse, la creazione di tabelle e l'inclusione di immagini esterne. \lipsum[1]
\end{abstract}

\section{Introduzione a \LaTeX}
\LaTeX{} è un sistema di preparazione di documenti di alta qualità, ampiamente utilizzato in ambito accademico, scientifico e tecnico per la produzione di articoli, tesi, libri e presentazioni. A differenza dei word processor WYSIWYG (What You See Is What You Get) come Microsoft Word, \LaTeX{} si basa su un approccio WYSIWYM (What You See Is What You Mean), in cui l'autore si concentra sulla struttura e sul contenuto del documento, lasciando al sistema il compito di gestire l'impaginazione e la formattazione.

\subsection{Struttura del Documento}
Un documento \LaTeX{} è tipicamente suddiviso in due parti principali: il \textbf{preambolo} e il \textbf{corpo} del documento.
\begin{itemize}
    \item \textbf{Preambolo:} Qui si definisce la classe del documento (es. `article`, `report`, `book`), si caricano i pacchetti necessari per funzionalità aggiuntive (es. `amsmath` per la matematica, `graphicx` per le immagini) e si impostano i metadati come titolo e autore.
    \item \textbf{Corpo:} Questa è la parte dove viene scritto il contenuto effettivo del documento, racchiusa tra i comandi `\begin{document}` e `\end{document}`.
\end{itemize}

\lipsum[2-3]

\section{Elementi di Formattazione}
\LaTeX{} offre un controllo completo su ogni aspetto del testo e degli elementi grafici. Di seguito sono presentati alcuni esempi comuni.

\subsection{Formule Matematiche}
Una delle maggiori forze di \LaTeX{} è la sua capacità di comporre espressioni matematiche complesse con una qualità tipografica eccezionale. Le formule possono essere inserite direttamente nel testo (inline), come ad esempio $E=mc^2$, oppure come blocchi separati ed numerati.

Ad esempio, la formula risolutiva per le equazioni di secondo grado $ax^2 + bx + c = 0$ è:
\begin{equation}
    x_{1,2} = \frac{-b \pm \sqrt{b^2 - 4ac}}{2a}
    \label{eq:quadratica}
\end{equation}

Mentre l'identità di Eulero, considerata da molti una delle più belle equazioni della matematica, è:
\begin{equation}
    e^{i\pi} + 1 = 0
    \label{eq:eulero}
\end{equation}

In questo contesto, è interessante notare che l'acqua, la cui formula chimica è $H_2O$, è un **composto inorganico** (classificato come ossido di diidrogeno) fondamentale per la vita.

\lipsum[4]

\subsection{Tabelle e Figure}
Creare tabelle e inserire figure è un'operazione semplice e versatile.

\subsubsection{Esempio di Tabella}
Le tabelle sono create usando l'ambiente `tabular`. È possibile specificare l'allineamento delle colonne e aggiungere bordi.

\begin{table}[h!]
\centering
\caption{Confronto delle proprietà di alcuni pianeti del Sistema Solare.}
\label{tab:pianeti}
\begin{tabular}{|l|c|c|c|}
\hline
\textbf{Pianeta} & \textbf{Diametro (km)} & \textbf{Massa ($10^{24}$ kg)} & \textbf{Satelliti} \\
\hline
Terra   & 12,742 & 5.97 & 1  \\
Marte   & 6,779  & 0.64 & 2  \\
Giove   & 139,820& 1898 & 95 \\
Saturno & 116,460& 568  & 146\\
\hline
\end{tabular}
\end{table}

\lipsum[5]

\subsubsection{Esempio di Figura}
Per inserire un'immagine, è necessario usare il pacchetto `graphicx` e il comando `\includegraphics`. Assicurati che il file dell'immagine (in questo caso `esempio-immagine.png`) si trovi nella stessa cartella del tuo file `.tex`.

\begin{figure}[h!]
    \centering
    % Sostituisci "esempio-immagine.png" con il nome del tuo file immagine
    % \includegraphics[width=0.7\textwidth]{esempio-immagine.png} 
    % Poiché non posso includere un file, uso un riquadro segnaposto
    \fbox{\parbox[c][10cm][c]{0.7\textwidth}{\centering Immagine di esempio \\ (sostituire con \texttt{\textbackslash includegraphics})}}
    \caption{Questa è una didascalia di esempio per un'immagine. Le figure possono essere referenziate nel testo usando il loro `label`, ad esempio Figura~\ref{fig:esempio}.}
    \label{fig:esempio}
\end{figure}

\section{Conclusioni}
Questo documento ha fornito una breve ma funzionale panoramica delle capacità di \LaTeX. Dalla gestione della struttura logica del testo alla composizione di elementi complessi come formule e tabelle, \LaTeX{} si dimostra uno strumento potente e indispensabile per la scrittura scientifica e accademica.

\lipsum[6-7]


% ----- FINE DEL DOCUMENTO -----
\end{document}