\documentclass[a4paper, 11pt]{article}

% ----- PREAMBOLO -----
\usepackage[utf8]{inputenc}
\usepackage[italian]{babel}

% Pacchetti standard
\usepackage{amsmath}        % Per la matematica avanzata
\usepackage{graphicx}       % Per le immagini
\usepackage{geometry}       % Per i margini
\usepackage{lipsum}         % Per il testo segnaposto
\usepackage{booktabs}       % Per tabelle più eleganti
\usepackage[colorlinks=true, linkcolor=blue, urlcolor=blue, citecolor=red]{hyperref} % Per link cliccabili

% Impostazioni dei margini
\geometry{a4paper, margin=2.5cm}

% Informazioni sul documento
\title{Un Modello Teorico per l'Analisi di Dati Complessi}
\author{
    Mario Rossi\\
    \small Dipsica, Università di Padova \\
    \small \texttt{mario.rossi@email.it}
    \and
    Giulia Bianchi\\
    \small Dipartimento di Statistica, Università di Padova \\
    \small \texttt{giulia.bianchi@email.it}
}
\date{Ottobre 2025}


% ----- INIZIO DEL DOCUMENTO -----
\begin{document}

\maketitle

\begin{abstract}
    Questo articolo introduce un nuovo modello stocastico per l'analisi di serie temporali ad alta dimensionalità. Il nostro approccio si basa sui lavori fondamentali di Einstein sulla relatività \cite{einstein1905} e sulle tecniche di programmazione literate proposte da Knuth \cite{knuth1984}. Presentiamo la derivazione matematica del modello, la sua implementazione computazionale e i risultati ottenuti su un dataset di benchmark. I risultati mostrano un miglioramento significativo rispetto ai metodi esistenti.
    \lipsum[1]
\end{abstract}

\section{Introduzione}
L'analisi di dati complessi è una sfida centrale in molti campi scientifici. I metodi tradizionali spesso faticano a gestire la dimensionalità e le non linearità presenti nei dati reali \cite{lamport1994}. Il nostro lavoro mira a superare questi limiti introducendo un framework unificato. In questo studio, come solvente per la preparazione dei campioni, è stato utilizzato l'etanolo ($C_2H_5OH$), un noto **composto organico** appartenente alla classe degli alcoli, per le sue eccellenti proprietà miscibili.

\lipsum[2]

\section{Metodologia}
Il nostro modello si basa sull'integrazione di un processo stocastico definito dalla seguente equazione differenziale, nota come Equazione di Langevin:
\begin{equation}
    m \frac{d^2\mathbf{x}}{dt^2} = -\gamma \frac{d\mathbf{x}}{dt} + \boldsymbol{\eta}(t)
    \label{eq:langevin}
\end{equation}
dove $m$ è la massa, $\gamma$ è il coefficiente di attrito e $\boldsymbol{\eta}(t)$ è un termine di rumore bianco Gaussiano con media nulla.

Per la validazione del modello, abbiamo utilizzato il dataset pubblico disponibile presso il repository di dati scientifici \cite{dataset2023}.
\lipsum[3]

\section{Risultati}
I risultati dei nostri esperimenti sono riassunti nella Tabella~\ref{tab:risultati}. Abbiamo confrontato il nostro metodo ("Modello Proposto") con due approcci standard del settore ("Metodo A" e "Metodo B") in termini di errore quadratico medio (MSE).

\begin{table}[h!]
\centering
\caption{Confronto delle performance dei modelli in termini di MSE (valori più bassi sono migliori).}
\label{tab:risultati}
\begin{tabular}{@{}lccc@{}}
\toprule
\textbf{Dataset} & \textbf{Metodo A} & \textbf{Metodo B} & \textbf{Modello Proposto} \\
\midrule
Dataset 1 (Sintetico) & 0.154 & 0.132 & \textbf{0.098} \\
Dataset 2 (Reale)     & 0.281 & 0.295 & \textbf{0.213} \\
Dataset 3 (Reale)     & 0.210 & 0.199 & \textbf{0.175} \\
\bottomrule
\end{tabular}
\end{table}

Come si può notare, il nostro modello ottiene un errore significativamente inferiore su tutti i dataset analizzati. La Figura~\ref{fig:convergenza} mostra la curva di convergenza dell'algoritmo di addestramento.

\begin{figure}[h!]
    \centering
    \fbox{\parbox[c][8cm][c]{0.8\textwidth}{\centering Grafico della convergenza del modello \\ (sostituire con \texttt{\textbackslash includegraphics})}}
    \caption{Andamento dell'errore di validazione durante le epoche di addestramento del nostro modello.}
    \label{fig:convergenza}
\end{figure}

\lipsum[4]

\section{Conclusioni}
In questo lavoro abbiamo presentato un nuovo modello che dimostra performance all'avanguardia. Le future direzioni di ricerca includono l'estensione del modello a dati non stazionari e l'ottimizzazione dell'implementazione per l'esecuzione su hardware dedicato come le GPU.

% ----- SEZIONE BIBLIOGRAFIA -----
% Stile della bibliografia (es. plain, unsrt, alpha, abbrv)
\bibliographystyle{plain} 

% File .bib che contiene le fonti
\bibliography{riferimenti} 

\end{document}