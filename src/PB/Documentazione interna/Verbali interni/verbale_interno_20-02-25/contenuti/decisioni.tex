% Intestazione
\fancyhead[L]{4 \hspace{0.2cm} Decisioni} % Testo a sinistra

\section{Decisioni}

Durante la riunione sono state prese le seguenti decisioni:

\vspace{0.5cm}

\begin{table}[htbp]
    \centering
    \rowcolors{2}{lightgray}{white}
    \begin{tabular}{|c|p{0.8\textwidth}|}
        \hline
        \rowcolor[gray]{0.75}
        \textbf{Codice} & \textbf{Descrizione}\\
        \hline
        VI 29.1 & E' stato deciso di che i file da cui il chatbot ha generato la risposta saranno resi disponibili all'utente tramite
        link.\\ 
        \hline
        VI 29.2 & E' stato deciso che non ci sarà più il concetto di \emph{sessione}, ogni messaggio sarà uguale. Nel
        \emph{database relazionale}\textsubscript{\textbf{\textit{G}}} ci sarà dunque un'unica tabella \emph{Messaggio}.\\ 
        \hline
        VI 29.3 & Le domande per proseguire la conversazione verranno generate basandosi solo sull'ultima coppia domanda-risposta,
        non potendo più basarsi sulla sessione corrente.\\ 
        \hline
        VI 29.4 & E' stato deciso che il requisito delle domande per iniziare la conversazione non verrà soddisfatto, a causa della
        difficoltà nel capire quando si può parlare di "inizio" della conversazione, dopo la rimozione delle sessioni.\\ 
        \hline
        VI 29.5 & Seguendo le indicazioni del \emph{proponente}, non verranno scritte istruzioni particolari nel
        \emph{Manuale Utente}\textsubscript{\textbf{\textit{G}}} oltre alla sezione "Cosa chiedere e come chiederlo".\\ 
        \hline
        VI 29.6 & E' stato deciso che il numero di messaggi da caricare ad ogni refresh verrà stabilito empiricamente solo in fase
        di programmazione.\\ 
        \hline
        VI 29.7 & E' stato deciso che verrà modellato il \emph{caso d'uso}\textsubscript{\textbf{\textit{G}}} per il caricamento
        dei vecchi messaggi con lo scroll dello schermo non appena sarà stato deciso il numero di messaggi da recuperare. In tale
        occasione, verrà modellato anche l'extend per il caso di fallimento del recupero dei messaggi.\\ 
        \hline
        VI 29.8 & E' stato stabilito di sistemare i diagrammi modellati nella progettazione per adattarli alle nuove istruzioni del
        proponente.\\ 
        \hline
        VI 29.9 & E' stato deciso che manderemo ad \emph{AzzurroDigitale} la sezione del Manuale Utente che fornisce le istruzioni su
        cosa chiedere al chatbot e come chiederlo.\\ 
        \hline
        VI 29.10 & E' stato deciso di non svolgere la verifica di fine sprint, in modo da poter fare subito la pull request nel ramo
        main, così da poter verificare il corretto aggiornamento dell'automazione.\\ 
        \hline
        VI 29.11 & E' stato deciso che Davide si occuperà di redigere il verbale interno.\\ 
        \hline
        VI 29.12 & E' stato deciso che Riccardo chiuderà su \emph{Jira}\textsubscript{\textbf{\textit{G}}} la sprint appena trascorsa
        e si segnerà le percentuali di completamento delle \emph{issues}\textsubscript{\textbf{\textit{G}}} per il cruscotto di
        qualità.\\ 
        \hline
        VI 29.13 & E' stato deviso che Riccardo invierà il verbale esterno ad \emph{AzzurroDigitale}.\\ 
        \hline
    \end{tabular}
\end{table}