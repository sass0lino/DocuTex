% Intestazione
\fancyhead[L]{B \hspace{0.2cm} Standard di qualità ISO/IEC 9126} % Testo a sinistra

\pagenumbering{roman} % Fa ripartire la numerazione romana delle pagine da I


\section{Standard di qualità ISO/IEC 9126}
\label{sec:standard_qualità_iso-iec_9126}

Lo standard per la valutazione della qualità che il gruppo \emph{SWEg Labs} ha adottato è l’\emph{ISO/IEC 9126}\textsubscript{\textit{\textbf{G}}}, 
il quale definisce diverse caratteristiche per garantire che il prodotto finale soddisfi
le esigenze dell’utente e del destinatario finale.

I criteri per la valutazione e misurazione della qualità si basano sui seguenti punti:
\begin{itemize}
    \item Funzionalità;
    \item Affidabilità;
    \item Usabilità;
    \item \emph{Efficienza}\textsubscript{\textit{\textbf{G}}};
    \item Manutenibilità;
    \item Portabilità.
\end{itemize}


\subsection{Funzionalità}

Questa caratteristica fa riferimento alle prestazioni del software rispetto alle specifiche funzionali 
previste, cioè la capacità del \emph{prodotto software}\textsubscript{\textit{\textbf{G}}} di fornire funzionalità che soddisfino
i requisiti stabiliti (o implicitamente dedotti). La Funzionalità si articola nelle seguenti
sotto-caratteristiche:
\begin{itemize}
    \item \textbf{Adeguatezza}: capacità del software di fornire un insieme di funzioni che soddisfino i requisiti specificati e le esigenze dell’utente;
    \item \textbf{Accuratezza}: capacità del software di fornire i risultati attesi con il grado di precisione richiesto;
    \item \textbf{Interoperabilità}: capacità del software di interagire con uno o più sistemi specificati;
    \item \textbf{Sicurezza}: capacità del software di proteggere informazioni e dati da accessi non autorizzati;
    \item \textbf{Aderenza alla funzionalità}: capacità del software di aderire a standard, convenzioni e regolamentazioni specificate.
\end{itemize}


\subsection{Affidabilità}
Questa caratteristica si riferisce alla capacità del software di operare con affidabilità nel corso
del tempo, cercando di ridurre il più possibile interruzioni o errori in circostanze specifiche.
L’affidabilità si compone delle seguenti sotto-caratteristiche:
\begin{itemize}
    \item \textbf{Maturità}: capacità del software di evitare errori, malfunzionamenti o guasti in condizioni specificate;
    \item \textbf{Tolleranza agli errori}: capacità del software di mantenere un livello specificato di prestazioni in caso di errori software o di violazioni della sua specifica;
    \item \textbf{Recuperabilità}: capacità del software di ripristinare il livello appropriato di prestazioni e di recupero dei dati in caso di malfunzionamenti;
    \item \textbf{Aderenza all’affidabilità}: capacità del software di aderire a standard, convenzioni e regolamentazioni specificate.
\end{itemize}


\subsection{Usabilità}
L’usabilità è un aspetto cruciale che influenza l’esperienza dell’utente nell’interazione con il
software. Le sotto-caratteristiche di usabilità delineate nell’\emph{ISO/IEC 9126}\textsubscript{\textit{\textbf{G}}} contribuiscono a
valutare l’\emph{efficacia}\textsubscript{\textit{\textbf{G}}}, l’\emph{efficienza}\textsubscript{\textit{\textbf{G}}} e la soddisfazione dell’utente. 
Queste sotto-caratteristiche sono:
\begin{itemize}
    \item \textbf{Comprensibilità}: la comprensibilità indica la facilità con cui gli utenti possono capire
    e comprendere il funzionamento del software. Un’interfaccia chiara e intuitiva favorisce
    la rapida comprensione delle funzionalità, riducendo la curva di apprendimento per gli
    utenti;
    \item \textbf{Curva d’apprendimento}: riflette la facilità con cui gli utenti possono imparare ad
    utilizzare il software. Un software con una buona curva d’apprendimento consente agli
    utenti di acquisire familiarità con le sue funzionalità in modo rapido ed efficiente;
    \item \textbf{Operabilità}: l’operabilità valuta la facilità con cui gli utenti possono interagire
    con il software per eseguire le operazioni desiderate. Un’interfaccia ben progettata
    e accessibile contribuisce a un’esperienza utente più efficiente e soddisfacente;
    \item \textbf{Apparenza}: l’apparenza si riferisce all’aspetto visivo del software e alla sua capacità 
    di suscitare interesse ed essere piacevole per gli utenti. Un'interfaccia accattivante 
    può migliorare la percezione complessiva del software e influenzare positivamente l’esperienza utente;
    \item \textbf{Conformità}: la conformità riguarda la coerenza del software rispetto alle convenzioni
    di usabilità e ai principi di design. Un’applicazione conforme alle \emph{best practices}\textsubscript{\textit{\textbf{G}}} di
    usabilità tende a fornire un’esperienza più uniforme e prevedibile agli utenti.
\end{itemize}


\subsection{Efficienza}
L’efficienza rappresenta la capacità del software di eseguire le proprie funzioni in modo
rapido ed economico, utilizzando in modo ottimale le risorse disponibili. L’efficienza la
possiamo individuare mediante:
\begin{itemize}
    \item \textbf{Comportamento rispetto al tempo}: questa sotto-caratteristica valuta la tempestività 
    con cui il software risponde alle richieste degli utenti. Un’applicazione efficiente
    è in grado di fornire risposte rapide, riducendo al minimo i tempi di attesa;
    \item \textbf{Utilizzo delle risorse}: l’efficienza nelle risorse misura quanto il software sia in grado
    di utilizzare in modo ottimale le risorse di sistema, come CPU, memoria e spazio di
    archiviazione. Un’applicazione efficiente è in grado di eseguire le proprie funzioni senza
    spreco eccessivo di risorse, garantendo prestazioni elevate e affidabili.
\end{itemize}


\subsection{Manutenibilità}
La manutenibilità è una caratteristica chiave nel valutare la qualità del software, concentrandosi 
sulla facilità con cui è possibile apportare modifiche al sistema, correggere errori e
migliorare le prestazioni nel tempo. Per permettere una facile manutenzione del software,
alcuni aspetti fondamentali da mantenere in considerazione sono:
\begin{itemize}
    \item \textbf{Analizzabilità}: l’analizzabilità valuta quanto sia agevole comprendere 
    la struttura del codice sorgente e individuare eventuali errori. Un software che già presenta
    un’approfondita Analisi dei Requisiti semplifica il processo di ispezione e correzione,
    accelerando le attività di manutenzione;
    \item \textbf{Modificabilità}: questa sotto-caratteristica riflette la facilità con cui è possibile apportare 
    modifiche al software, aggiungere nuove funzionalità o adattarlo a nuovi requisiti.
    Un sistema altamente modificabile è più flessibile e reattivo agli aggiornamenti;
    \item \textbf{Stabilità}: la stabilità indica la capacità del software di mantenere la coerenza delle 
    prestazioni anche dopo l’introduzione di modifiche. Un’applicazione stabile minimizza 
    gli effetti collaterali delle modifiche, garantendo che le nuove funzionalità non
    compromettano l’integrità del sistema;
    \item \textbf{Testabilità}: la testabilità permette di misurare quanto sia agevole
    verificare e validare le modifiche apportate al software. Un’applicazione con un’elevata
    copertura dei test semplifica il processo di identificazione e risoluzione di problemi,
    facilitando il mantenimento del software nel tempo.
\end{itemize}


\subsection{Portabilità}
La portabilità si riferisce alla capacità del software di essere trasferito o adattato facilmente 
a diversi ambienti, sistemi operativi o architetture senza perdere le sue funzionalità e
prestazioni. Queste capacità possono essere individuate come:
\begin{itemize}
    \item \textbf{Adattabilità}: l’adattabilità indica quanto il software può essere modificato per adattarsi 
    a nuovi ambienti o requisiti. Un’applicazione adattabile è in grado di operare senza
    problemi su diverse piattaforme, consentendo una maggiore flessibilità operativa;
    \item \textbf{Installabilità}: questa sotto-caratteristica riflette la facilità con cui il software può
    essere installato in diversi ambienti. Un’applicazione facilmente installabile semplifica
    il processo di distribuzione e implementazione su varie configurazioni di sistema;
    \item \textbf{Conformità}: la conformità si riferisce alla capacità del software di rispettare gli
    standard e i protocolli di interoperabilità. Un’applicazione conforme è in grado di
    interagire in modo coerente con altri sistemi e applicazioni, facilitando l’integrazione
    in ambienti eterogenei;
    \item \textbf{Sostituibilità}: la sostituibilità valuta quanto sia agevole sostituire il software con
    un’altra soluzione equivalente. Un’applicazione sostituibile agevola il processo di migrazione verso 
    nuove tecnologie o soluzioni senza eccessivi sforzi di adattamento.
\end{itemize}