% Intestazione
\fancyhead[L]{3 \hspace{0.2cm} Diario della riunione} % Testo a sinistra

\section{Diario della riunione}

\begin{itemize}
    \item Sono stati presentati al \emph{proponente} i nostri obiettivi di funzionalità per l'\emph{MVP} aggiornati rispetto all'ultimo incontro, ovvero: 
        \begin{itemize}
            \item Salvataggio e recupero dello storico di sessione;
            \item Proporre domande per proseguire la conversazione;
            \item Visualizzazione del badge dell’esito dell’aggiornamento automatico;
            \item Visualizzazione di un txt con i log dell’aggiornamento automatico;
            \item Migliorare l’aggiornamento automatico per gestire l’eliminazione di file;
            \item Migliorare l’aggiornamento automatico dei file che subiscono modifiche;
            \item Visualizzazione dei link dei file da cui il bot ha preso la risposta;
            \item Pulsante di copia messaggio e pulsante di copia snippet di codice.
        \end{itemize}
    \item Sono state presentate al \emph{proponente} le funzionalità che non verranno implementate nell'\emph{MVP}, ovvero:
        \begin{itemize}
            \item Proporre domande per iniziare la conversazione;
            \item Integrazione delle \emph{API}\textsubscript{\textbf{\textit{G}}} verso
            \emph{Telegram}\textsubscript{\textbf{\textit{G}}} e \emph{Slack}\textsubscript{\textbf{\textit{G}}}.
        \end{itemize}

    \item Abbiamo chiesto consiglio al \emph{proponente} riguardo la gestione della modifica dei documenti nel \emph{database vettoriale} \emph{Chroma}, e ci è stato consigliato di provare a prelevare la data di ultima modifica del documento dalla sua piattaforma di provenienza (es.: \emph{GitHub}\textsubscript{\textbf{\textit{G}}}) e quindi di confrontarla con la data di inserimento del documento nel \emph{database vettoriale}: se la data di modifica è più recente, allora il documento deve essere aggiornato, altrimenti no;
    \item Abbiamo chiesto consiglio al \emph{proponente} riguardo la gestione dell'eliminazione dei documenti nel \emph{database vettoriale} \emph{Chroma}, e ci è stato consigliato, per maggiore efficienza, di prelevare solamemente gli id dei documenti presenti in \emph{Chroma} per confrontarli con gli id dei documenti prelevati dalle piattaforme, senza prelevare i documenti interi, come avevamo implementato inizialmente;
    \item Abbiamo chiesto consiglio al \emph{proponente} riguardo la gestione della ricreazione della connessione verso i \emph{database}, e ci è stato riferito che nel mondo aziendale si utilizzano dei \emph{framework}\textsubscript{\textbf{\textit{G}}} che gestiscono la connessione in modo automatico, e che quindi non è necessario gestire manualmente la ricreazione della connessione. Per il nostro progetto, la questione è stata valutata come non necessaria;
    \item Abbiamo chiesto consiglio al \emph{proponente} riguardo la gestione dell'errore di salvataggio dei log nel \emph{database relazionale} \emph{Postgres} da parte del \emph{cron}\textsubscript{\textbf{\textit{G}}}, in particolare a causa del fatto che, in caso di tale fallimento, \emph{Angular}\textsubscript{\textbf{\textit{G}}} potrebbe recuperare dal database i log di due volte prima, e quindi non i log più recenti. Il \emph{proponente} ci ha esposto tre possibilità di gestione del consenso dei dati tra quanto salva il \emph{cron} e quanto recupera \emph{Angular}:
        \begin{itemize}
            \item \emph{WebSocket}\textsubscript{\textbf{\textit{G}}}: il \emph{cron} invia i log ad \emph{Angular} tramite \emph{WebSocket} ed \emph{Angular} così risulta aggiornato in tempo reale senza passare per un database. Tuttavia, questa soluzione è stata valutata come troppo complessa per il nostro progetto;
            \item \emph{Polling REST}\textsubscript{\textbf{\textit{G}}}: come già avviene attualmente, il \emph{cron} ed \emph{Angular} agiscono in modo autonomo, cioè \emph{Angular} recupera i log dal \emph{database} ogni tot tempo; a questo punto, utilizzando particolari framework, è possibile garantire il consenso dei dati tra i due;
            \item Verificare lo stato dell'aggiornamento automatico solamente al momento di apertura o refresh della scheda del \emph{browser}\textsubscript{\textbf{\textit{G}}}: è stato infatti constatato assieme al proponente che è molto improbabile che l'utente mantenga aperta la scheda del \emph{browser} per un tempo sufficientemente lungo da causare problemi di sincronizzazione tra il \emph{cron} ed \emph{Angular}, e quindi è possibile anche questa terza opzione perchè \emph{Angular} ottenga l'esito dell'ultimo aggiornamento.
        \end{itemize} 
    Mentre la gestione degli errori è stata valutata come requisito non necessario, le ultime due possibilità di sincronizzazione del badge del \emph{frontend}\textsubscript{\textbf{\textit{G}}} sono state valutate come equivalenti per il proponente, e quindi la scelta è stata lasciata a noi;

    \item Abbiamo mostrato al proponente il file di testo dove sono stati stampati i log nella nostra implementazione attuale, e ci è stato detto che la visualizzazione è accettabile. Inoltre, ci è stata lasciata carta bianca per il salvataggio dei log come pila o come coda;
    \item Abbiamo mostrato al proponente il nostro secondo file per il salvataggio dei log, cron.log, spiegando qual è il rapporto di quest'ultimo con il file di testo: il file txt contiene il log consuntivo di ogni tentativo di aggiornamento automatico, mentre il file cron.log consente di monitonare l'operato del cron, poichè trascrive tutto ciò che viene stampato nello standard output della console durante la sua esecuzione. Il proponente ha approvato questa soluzione di spartizione delle responsabilità tra i due file;
    \item Abbiamo chiesto al proponente come gestire il fatto che, mentre il database vettoriale sta venendo aggiornato, il chatbot non è in grado di rispondere, e ci è stato consigliato di riportare il fenomeno nel documento \emph{Manuale Utente}\textsubscript{\textbf{\textit{G}}};
    \item Abbiamo mostrato al proponente la prima bozza del \emph{frontend}, e ci è stato detto che la grafica è accettabile, e che possiamo procedere seguendo tale direzione. Per i link ai documenti da cui è stata ricavata la risposta, è stata approvata la proposta di fare un riquadro in fondo al messaggio, con un link cliccabile per ogni documento. Ci è stato inoltre approvato l'inserimento della data e ora di invio nei messaggi;
    \item Abbiamo riportato al \emph{proponente} i risultati attuali per quanto riguarda i test: abbiamo fino ad ora svolto i \emph{test di integrazione}\textsubscript{\textbf{\textit{G}}}, che ci hanno permesso di ottenere una \emph{coverage} delle righe del 60\% per il \emph{backend}\textsubscript{\textbf{\textit{G}}}, e del 70\% per il \emph{frontend}. Abbiamo poi chiesto al proponente quanta coverage sia richiesta, e quali tipologie di coverage siano necessarie. Il proponente ci ha risposto che ci farà sapere in seguito. Abbiamo inoltre comunicato che nella prossima \emph{sprint} ci occuperemo di sviluppare i \emph{test di unità}\textsubscript{\textbf{\textit{G}}};
    \item Abbiamo chiesto al \emph{proponente} a quanto ammonta in termini di denaro il nostro utilizzo della \emph{API Key}\textsubscript{\textbf{\textit{G}}} di \emph{OpenAI}\textsubscript{\textbf{\textit{G}}} allo stato attuale, e ci è stato risposto che il consumo finora è stato abbastanza basso, e che quindi possiamo continuare ad utilizzarla in modo normale per le nostre necessità;
    \item Abbiamo infine riportato al \emph{proponente} che siamo avanzati poco con la documentazione a causa del fatto che nella scorsa sprint abbiamo puntato quasi esclusivamente sulla progettazione e sulla programmazione, e abbiamo promesso che nella prossima \emph{sprint} riprenderemo i lavori con la parte documentale: in particolare, riporteremo i \emph{Diagrammi delle classi}\textsubscript{\textbf{\textit{G}}} nel documento di \emph{Specifica Tecnica}\textsubscript{\textbf{\textit{G}}}.

\end{itemize}