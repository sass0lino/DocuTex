% Intestazione
\fancyhead[L]{3 \hspace{0.2cm} Diario della riunione} % Testo a sinistra

\section{Diario della riunione}

\begin{itemize}
    \item Sono stati presentati al \emph{proponente} i \emph{test di accettazione} scritti per l'MVP. Il proponente li ha letti e ha suggerito alcune modifiche;
    \item E' stata svolta la \emph{demo} dell'MVP, durante la quale sono stati approvati tutti i test di accettazione proposti, tranne TA9 ("Verificare che l'utente possa visualizzare una lista di domande ideali per poter iniziare una conversazione"), per il quale era già stata concordata la non necessità con il proponente nella riunione precedente;
    \item Il proponente ha posto alcune domande sul prodotto:
    \begin{itemize}
        \item Ha chiesto se è possibile, nel caso in cui una piattaforma non sia raggiungibile, e quindi l'aggiornamento automatico del \emph{database vettoriale}\textsubscript{\textbf{\textit{G}}} non avvenga  in modo completo, lanciare manualmente l'aggiornamento solo per quell'unica piattaforma che non era stata raggiungibile. La risposta è stata negativa, poichè lo script di aggiornamento è unico e non prevede la possibilità di aggiornare solo una piattaforma;
        \item Ha chiesto se è possibile porre al chatbot domande e ricevere risposte anche in una lingua diversa dall'italiano. La risposta è stata negativa: il vincolo della sola lingua italiana è, infatti, scritto nell'header che fornisce il contesto all'\emph{LLM}\textsubscript{\textbf{\textit{G}}};
        \item Ha chiesto se il chatbot può rispondere a domande che prevedano il coinvolgimento di documenti provenienti da più di una singola piattaforma. La risposta è stata affermativa, ed è stata mostrata una risposta di esempio che conteneva informazioni provenienti sia da \emph{GitHub}\textsubscript{\textbf{\textit{G}}} che da
        \emph{Jira}\textsubscript{\textbf{\textit{G}}};
        \item Ha chiesto se la libreria \emph{requests} di \emph{Python}\textsubscript{\textbf{\textit{G}}}, utilizzata per le chiamate \emph{API}\textsubscript{\textbf{\textit{G}}} verso \emph{Jira} e verso \emph{Confluence}\textsubscript{\textbf{\textit{G}}} possiede un limite massimo nel numero di richieste che può effettuare. Il gruppo ha risposto che non è a conoscenza di un limite massimo, e che si informerà il prima possibile per poter rispondere con certezza una volta terminato l'incontro.
    \end{itemize}
    \item Sono state descritte le funzionalità soddisfatte nell'MVP in aggiunta a quelle già implementate nel PoC, che corrispondono a tutte le funzionalità discusse e concordate assieme negli scorsi incontri esterni con il proponente;
    \item E' stata presentata la \emph{coverage} ottenuta con i \emph{test di unità} e i \emph{test di integrazione}, che è risultata essere del 90\% sia per il \emph{backend}\textsubscript{\textbf{\textit{G}}} sia per il \emph{frontend}\textsubscript{\textbf{\textit{G}}}, che sono entrambi valori sufficienti rispetto alla soglia di accettazione del 75\% concordata con il proponente;
    \item E' stata chiesta conferma della non necessità di implementare il requisito RDF33 ("Nel caso non ci siano altri messaggi nel database, l'utente deve visualizzare un messaggio che comunica che non ci sono altri messaggi da visualizzare") per l'MVP, siccome richiederebbe di creare un intero \emph{componente}\textsubscript{\textbf{\textit{G}}} aggiuntivo e i migliori chatbot in circolazione non lo possiedono; il proponente ha confermato che non è necessario implementarlo;
    \item E' stata chiesta la possibilità di ottenere la firma del verbale esterno della riunione già giovedì 20 marzo, il giorno successivo, in quanto necessaria per poter chiedere la candidatura al primo colloquio della \emph{PB}; il proponente ha confermato che sarà possibile;
    \item E' stato aggiornato il proponente sullo stato della documentazione. In particolare, è stato riferito che è stato completato il \emph{Manuale Sviluppatore}\textsubscript{\textbf{\textit{G}}} richiesto via \emph{Discord}\textsubscript{\textbf{\textit{G}}}, e che sono in corso le attività di verifica del resto della documentazione per preparare la \emph{PB};
    \item E' stato chiesto al proponente come desidera che sia reperibile l'applicativo "BuddyBot", e il proponente ha risposto che è sufficiente il \emph{repository}\textsubscript{\textbf{\textit{G}}} \emph{GitHub}\textsubscript{\textbf{\textit{G}}} già presente, e che sarà molto importante il README lì presentato per la fornitura delle istruzioni di installazione e avvio;
    \item E' stato chiesto al proponente di confermare il materiale che è richiesto consegnargli, che è stato confermato essere:
    \begin{itemize}
        \item Documento di analisi funzionale e tecnica;
        \item Test;
        \item Verbale di collaudo;
        \item \emph{Manuale Utente}\textsubscript{\textbf{\textit{G}}};
        \item Link al repository dell'MVP;
        \item Link al sito \emph{GitHub Pages}\textsubscript{\textbf{\textit{G}}} della documentazione;
        \item Manuale Sviluppatore.
    \end{itemize}
    Inoltre, è stato chiesto come e quando fornire il materiale richiesto, e il proponente ha risposto che è sufficiente inviare tutto via email successivamente all'ultimo incontro della PB, quindi al termine del progetto, e che è già possibile fornire il link al repository dell'MVP via Discord;
    \item Il proponente ha proposto al gruppo di svolgere un incontro in presenza nella loro sede aziendale, assieme all'altro gruppo che sta svolgendo lo stesso capitolato, per presentare il progetto a tutti i dipendenti di \emph{AzzurroDigitale}. Il gruppo ha accettato la proposta, dunque il proponente si occuperà di organizzare l'incontro dopo che entrambi i gruppi avranno terminato la PB.

\end{itemize}