% Intestazione
\fancyhead[L]{4 \hspace{0.2cm} Decisioni} % Testo a sinistra


\section{Decisioni}

Durante la riunione sono state prese le seguenti decisioni:

\vspace{0.5cm}

\begin{table}[htbp]
    \centering
    \rowcolors{2}{lightgray}{white}
    \begin{tabular}{|c|p{0.8\textwidth}|}
        \hline
        \rowcolor[gray]{0.75}
        \multicolumn{1}{|c|}{\textbf{Codice}} & \multicolumn{1}{|c|}{\textbf{Descrizione}}\\
        \hline
        VE 4.1 & È stato deciso di attendere la risposta di \emph{AzzurroDigitale} per quanto riguarda il supporto economico
        per la API Key di OpenAI per GPT-4o, prima di sostenere eventuali ulteriori esplorazioni su altri LLM. \\
        \hline
        VE 4.2 & È stato stabilito che nel PoC verrà visualizzata la risposta dell'LLM solamente da terminale.
        Cioè, nella RTB è stato escluso lo sviluppo front-end. \\
        \hline
        VE 4.3 & È stato deciso che il chatbot accederà a GitHub autenticato tramite il token della organization \emph{SWEg Labs},
        per ottenere il largo limite di 2000 richieste autenticate all'ora ed evitare lo stretto limite di 60 richieste non autenticate all'ora. \\
        \hline
        VE 4.4 & È stato deciso che, in Confluence, per fare da contesto al chatbot, verranno caricati documenti estranei alla documentazione del progetto didattico
        (distribuita in PDF), data l'impossibilità di caricare PDF in modo diretto nella piattaforma. \\
        \hline
    \end{tabular}
\end{table}
