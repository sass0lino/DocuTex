
\fancyhead[L]{4 \hspace{0.2cm} Requisiti} % Testo a sinistra

\section{Requisiti}
\label{sec:Requisiti}   

\subsection{Requisiti funzionali}
\label{sec:requisiti_funzionali}
Questa sezione delinea i requisiti funzionali del sistema. Gli obiettivi e le azioni chiave che
l'utente deve essere in grado di compiere sono presentati in modo chiaro, fornendo una base
solida per la \emph{progettazione}\textsubscript{\textit{\textbf{G}}} del sistema.
\begin{table}[h!]
    \centering
    \renewcommand{\arraystretch}{1.6} % Per aumentare l'altezza delle righe
    \begin{tabularx}{\textwidth}{|p{2cm}|p{3cm}|X|p{4cm}|} \hline
    \rowcolor[HTML]{FFD700} 
    \textbf{Codice} & \textbf{Rilevanza} & \textbf{Descrizione} & \textbf{Fonti} \\ \hline
    ROF1 & Obbligatorio & L'utente deve poter inserire un'interrogazione in linguaggio naturale nel sistema. & \bulhyperlink{UC1}{UC1} \\ \hline
    ROF2 & Obbligatorio & Quando l'interrogazione viene invata al sistema, deve essere generata una risposta. & \bulhyperlink{UC2}{UC2} \\ \hline
    ROF3 & Obbligatorio & Nel caso il sistema fallisca nel generare una risposta per via di un problema interno, deve far visualizzare all'utente un messaggio di errore, chiedendo di riprovare più tardi. & \bulhyperlink{UC2}{UC2}, \bulhyperlink{UC3}{UC3} \\ \hline
    ROF4 & Obbligatorio & Nel caso in cui l'utente inserisca un'interrogazione che non riguarda i contenuti del database associato, il sistema deve rispondere all'utente che la domanda inserita è fuori contesto. & \bulhyperlink{UC2}{UC2}, \bulhyperlink{UC4}{UC4} \\ \hline
    ROF5 & Obbligatorio & Nel caso il sistema non riesca a trovare le informazioni richieste dall'utente nonostante siano correlate al contesto, deve rispondere all'utente spiegando la mancanza dell'informazione richiesta. & \bulhyperlink{UC2}{UC2}, \bulhyperlink{UC5}{UC5} \\ \hline
    ROF6 & Desiderabile & L'utente deve poter visualizzare i file da cui il sistema ha preso i dati per la risposta. &\bulhyperlink{UC2.1}{UC2.1} \\ \hline
    RDF7 & Desiderabile & Nel caso l'utente desideri accedere ai file utilizzati per generare la risposta ma questo non sia possibile, deve essere visualizzato un messaggio di errore. &\bulhyperlink{UC14}{UC14} \\ \hline
    \end{tabularx}
    \end{table}

    \vspace{0.5cm}
    \newpage
    % Seconda parte della tabella
    \begin{table}[h!]
    \renewcommand{\arraystretch}{1.6} % Per aumentare l'altezza delle righe
    \begin{tabularx}{\textwidth}{|p{2cm}|p{3cm}|X|p{4cm}|} \hline
    \rowcolor[HTML]{FFD700} 
    \textbf{Codice} & \textbf{Rilevanza} & \textbf{Descrizione} & \textbf{Fonti} \\ \hline
        RZF8 & Opzionale & Deve essere presente un pulsante al cui click la risposta del chatbot viene copiata nel dispositivo dell'utente. & \bulhyperlink{UC6}{UC6} \\ \hline
        RDF9 & Desiderabile & Nel caso la risposta contenga uno snippet di codice, deve essere presente un pulsante che permetta di copiare il singolo snippet nel dispositivo dell'utente. & \bulhyperlink{UC7}{UC7} \\ \hline
        RDF10 & Desiderabile & Deve essere presente un sistema di archiviazione delle domande e delle risposte in un database relazionale. & \bulhyperlink{UC8}{UC8} \\ \hline
        RDF11 & Desiderabile & L'utente deve poter visualizzare lo storico della chat, recuperato dal database relazionale. & \bulhyperlink{UC8}{UC8}, \bulhyperlink{UC8.1}{UC8.1} \\ \hline
        RDF12 & Desiderabile & Nel caso il sistema fallisca nel recuperare lo storico della chat, deve essere fatto visualizzare un messaggio di errore all'utente spiegando che non è stato possibile recuperare lo storico. & \bulhyperlink{UC9}{UC9} \\ \hline
        ROF13 & Obbligatorio & Uno scheduler deve collegarsi al sistema e periodicamente aggiornare il database vettoriale con i dati più recenti. & \bulhyperlink{UC10}{UC10} \\ \hline
        ROF14 & Obbligatorio & La risposta deve essere generata prendendo in considerazione i dati di contesto provenienti da GitHub,Jira e Confluence & \bulhyperlink{UC10}{UC10}, Capitolato \\ \hline
        ROF15 & Obbligatorio & Il sistema deve poter convertire i dati ottenuti in formato vettoriale. & Verbale interno \\ \hline
        ROF16 & Obbligatorio & Il sistema deve poter aggiornare il database vettoriale con i nuovi dati ottenuti. & Verbale interno \\ \hline
        RZF17 & Opzionale & Se la conversazione non è ancora avviata l'utente deve poter visualizzare e selezionare alcune domande di partenza proposte. & \bulhyperlink{UC11}{UC11},\bulhyperlink{UC11.1}{UC11.1} \\ \hline
    \end{tabularx}

    \caption{Requisiti funzionali}
    \end{table}

    \vspace{0.5cm}
    \newpage
    % Terza parte della tabella
    \begin{table}[h!]
    \renewcommand{\arraystretch}{1.6} % Per aumentare l'altezza delle righe
    \begin{tabularx}{\textwidth}{|p{2cm}|p{3cm}|X|p{4cm}|} \hline
        \rowcolor[HTML]{FFD700} 
        \textbf{Codice} & \textbf{Rilevanza} & \textbf{Descrizione} & \textbf{Fonti} \\ \hline
            RZF18 & Opzionale & Dopo la visualizzazione di una risposta, all'utente devono venire suggerite alcune interrogazioni che è possibile porre al sistema per continuare la conversazione. & \bulhyperlink{UC12}{UC12},\bulhyperlink{UC12.1}{UC12.1} \\ \hline
            RZF19 & Opzionale & Nel caso il sistema vada in errore nel tentativo di proporre altre domande per proseguire la conversazione, deve venire mostrato un messaggio che comunica l'errore all'utente e invita a fare altre domande. & \bulhyperlink{UC13}{UC13} \\ \hline
            RZF20 & Opzionale & Il sistema deve registrare data e ora degli aggiornamenti del database vettoriale, in modo da poter scrivere un log di aggiornamento. & \bulhyperlink{UC15}{UC15} \\ \hline
            RZF21 & Opzionale & Il sistema deve comunicare se il database vettoriale a cui vengono poste le interrogazioni è aggiornato o meno. & \bulhyperlink{UC15}{UC15}, \bulhyperlink{UC16}{UC16}, \bulhyperlink{UC17}{UC17} \\ \hline
    \end{tabularx}

    \caption{Requisiti funzionali}
    \label{tab:Requisiti_funzionali}
\end{table}

\subsection{Requisiti qualitativi}
\label{sec:Requisiti_qualitativi}
I requisiti qualitativi del sistema sono trattati in questo sotto-capitolo. Questa sezione
delinea le specifiche qualitative che devono essere rispettate per garantire la qualità del
sistema.
\begin{table}[h!]
    \centering
    \renewcommand{\arraystretch}{1.6} % Per aumentare l'altezza delle righe
    \begin{tabularx}{\textwidth}{|>{\centering\arraybackslash}c|>{\centering\arraybackslash}c|>{\centering\arraybackslash}X|>{\centering\arraybackslash}p{3cm}|} \hline
    \rowcolor[HTML]{FFD700} 
    \textbf{Codice} & \textbf{Rilevanza} & \textbf{Descrizione} & \textbf{Fonti} \\ \hline
    ROQ1 & Obbligatorio & Devono essere rispettate tutte le norme definite in \emph{Norme di Progetto}\textsubscript{\textbf{G}}. & Verbale Interno \\ \hline
    ROQ2 & Obbligatorio & Devono essere rispettate le \emph{metriche}\textsubscript{\textbf{G}} e i vincoli definiti in \emph{Piano di Qualifica}\textsubscript{\textbf{G}}. & Verbale Interno \\ \hline
    ROQ3 & Obbligatorio & Deve essere fornito al \emph{proponente} il codice sorgente in un \emph{repository}\textsubscript{\textbf{G}} \emph{GitHub}\textsubscript{\textbf{G}}. & Capitolato \\ \hline
    ROQ4 & Obbligatorio & Deve essere fornito il \emph{Manuale Utente}\textsubscript{\textbf{G}}. & Capitolato \\ \hline
    \end{tabularx}
    \caption{Requisiti qualitativi}
    \label{tab:Requisiti_qualitativi}
\end{table}


\newpage
\subsection{Requisiti di vincolo}
\label{sec:req_vincolo}
Qui sono presentati i requisiti di vincolo, che rappresentano le restrizioni e le condizioni
che devono essere soddisfatte durante lo sviluppo e l'implementazione del sistema. Questa
sezione fornisce le linee guida fondamentali che devono essere rispettate per garantire la
coerenza e l'\emph{efficienza}\textsubscript{\textit{\textbf{G}}} del prodotto.
\begin{table}[h!]
    \centering
    \renewcommand{\arraystretch}{1.6} % Per aumentare l'altezza delle righe
    \begin{tabularx}{\textwidth}{|>{\centering\arraybackslash}c|>{\centering\arraybackslash}c|>{\centering\arraybackslash}X|>{\centering\arraybackslash}p{3cm}|} \hline
    \rowcolor[HTML]{FFD700} 
    \textbf{Codice} & \textbf{Rilevanza} & \textbf{Descrizione} & \textbf{Fonti} \\ \hline
	ROV1 & Obbligatorio & L'applicazione deve garantire la compatibilità con la versione più recente di \emph{Google Chrome}\textsubscript{\textit{\textbf{G}}} al momento della \emph{demo}\textsubscript{\textit{\textbf{G}}}. & Verbale esterno \\ \hline
    ROV2 & Obbligatorio & Il sistema deve garantire piena integrazione con le \emph{API}\textsubscript{\textit{\textbf{G}}} di \emph{Confluence}\textsubscript{\textit{\textbf{G}}}. & Capitolato \\ \hline
    ROV3 & Obbligatorio & Il sistema deve garantire piena integrazione con le \emph{API}\textsubscript{\textit{\textbf{G}}} di \emph{Jira}\textsubscript{\textit{\textbf{G}}}. & Capitolato \\ \hline
    ROV4 & Obbligatorio & Il sistema deve garantire piena integrazione con le \emph{API}\textsubscript{\textit{\textbf{G}}} di \emph{GitHub}\textsubscript{\textit{\textbf{G}}}. & Capitolato \\ \hline
    \end{tabularx}
    \caption{Requisiti di vincolo}
    \label{tab:Requisiti_di_vincolo}
\end{table}

\subsection{Requisiti implementativi}
\label{sec:Requisiti_implementativi}
\begin{table}[h!]
    \centering
    \renewcommand{\arraystretch}{1.6} % Per aumentare l'altezza delle righe
    \begin{tabularx}{\textwidth}{|>{\centering\arraybackslash}c|>{\centering\arraybackslash}c|>{\centering\arraybackslash}X|>{\centering\arraybackslash}p{3cm}|} \hline
    \rowcolor[HTML]{FFD700} 
    \textbf{Codice} & \textbf{Rilevanza} & \textbf{Descrizione} & \textbf{Fonti} \\ \hline
    ROI1 & Obbligatorio & Il sistema deve processare i documenti che vengono caricati, creandone i loro \emph{embedding}. & Verbale interno\\ \hline
    ROI2 & Obbligatorio & Il sistema deve salvare, in modo persistente, il contenuto dei documenti caricati. & Verbale interno\\ \hline
    ROI3 & Obbligatorio & Il sistema deve salvare, in modo persistente, tutti i metadati dei documenti caricati. & Verbale interno\\ \hline
    ROI4 & Obbligatorio & Il sistema deve salvare, i commit e i file scritti in caratteri testuali di \emph{GitHub}, escludendo PDF o immagini. & Verbale esterno\\ \hline
    \end{tabularx}
    \caption{Requisiti implementativi}
    \label{tab:Requisiti_implementativi}
\end{table}

\subsection{Requisiti prestazionali}
\label{sec:req_prestazionali}
Trattandosi di una \emph{applicazione web}\textsubscript{\textit{\textbf{G}}}, i requisiti prestazionali saranno influenzati principalmente dalla connessione Internet, 
dalle prestazioni del dispositivo e dall'ottimizzazione specifica per il browser \emph{Google Chrome}, l'unico supportato ufficialmente.

\newpage
\subsection{Tracciamento}
\subsubsection{Fonte - Requisiti}
\label{sec:fonte_requisito}
\begin{table}[h!]
    \centering
    \renewcommand{\arraystretch}{1.6} % Per aumentare l'altezza delle righe
    \begin{tabularx}{0.8\textwidth}{|>{\centering\arraybackslash}p{2.8cm}|>{\centering\arraybackslash}X|} \hline
    \rowcolor[HTML]{FFD700} 
    \textbf{Fonte} & \textbf{Requisiti} \\ \hline
    Capitolato & ROQ3, ROQ4, ROQ5 \\ \hline
    Verbali interni & ROQ1, ROQ2, ROF15, ROF16  \\ \hline
    Verbali esterni & ROV1, ROV2, ROV3, ROV4, ROI1, ROI2, ROI3, ROI4\\ \hline
    \bulhyperlink{UC1}{UC1} & ROF1 \\ \hline
    \bulhyperlink{UC2}{UC2} & ROF2 \\ \hline
    \bulhyperlink{UC2}{UC2}, \bulhyperlink{UC3}{UC3} & ROF3 \\ \hline
    \bulhyperlink{UC2}{UC2}, \bulhyperlink{UC4}{UC4} & ROF4 \\ \hline
    \bulhyperlink{UC2}{UC2}, \bulhyperlink{UC5}{UC5} & ROF5 \\ \hline
    \bulhyperlink{UC2.1}{UC2.1} & ROF6 \\ \hline
    \bulhyperlink{UC14}{UC14} & RDF7 \\ \hline
    \bulhyperlink{UC6}{UC6} & RZF8 \\ \hline
    \bulhyperlink{UC7}{UC7} & RDF9 \\ \hline
    \bulhyperlink{UC8}{UC8} & RDF10 \\ \hline
    \bulhyperlink{UC8}{UC8}, \bulhyperlink{UC8.1}{UC8.1} & RDF11 \\ \hline
    \bulhyperlink{UC9}{UC9} & RDF12 \\ \hline
    \bulhyperlink{UC10}{UC10} & ROF13 \\ \hline
    \bulhyperlink{UC10}{UC10}, Capitolato & ROF14 \\ \hline
    \bulhyperlink{UC11}{UC11}, \bulhyperlink{UC11.1}{UC11.1} & RZF17 \\ \hline
    \bulhyperlink{UC12}{UC12}, \bulhyperlink{UC12.1}{UC12.1} & RZF18 \\ \hline
    \bulhyperlink{UC13}{UC13} & RZF19 \\ \hline
    \bulhyperlink{UC15}{UC15}& RZF20 \\ \hline
    \bulhyperlink{UC15}{UC15},\bulhyperlink{UC16}{UC16}, \bulhyperlink{UC17}{UC17} & RZF21 \\ \hline
    \end{tabularx}
    \caption{Tracciamento Fonte - Requisiti}
    \label{tab:Tracciamento_fonte_requisiti}
\end{table}


\newpage
\subsubsection{Requisito - Fonti}
\label{sec:requisito_fonte}
\begin{table}[h!]
    \centering
    \renewcommand{\arraystretch}{1.6} % Per aumentare l'altezza delle righe
    \begin{tabularx}{0.8\textwidth}{|>{\centering\arraybackslash}p{2.8cm}|>{\centering\arraybackslash}X|} \hline
    \rowcolor[HTML]{FFD700} 
    \textbf{Requisito} & \textbf{Fonti} \\ \hline
    ROF1 & \bulhyperlink{UC1}{UC1}\\ \hline
    ROF2 & \bulhyperlink{UC2}{UC2}\\ \hline
    ROF3 & \bulhyperlink{UC2}{UC2}, \bulhyperlink{UC3}{UC3}\\ \hline
    ROF4 & \bulhyperlink{UC2}{UC2}, \bulhyperlink{UC4}{UC4}\\ \hline
    ROF5 & \bulhyperlink{UC2}{UC2}, \bulhyperlink{UC5}{UC5}\\ \hline
    ROF6 & \bulhyperlink{UC2.1}{UC2.1}\\ \hline
    RDF7 & \bulhyperlink{UC14}{UC14} \\ \hline
    RZF8 & \bulhyperlink{UC6}{UC6}\\ \hline
    RDF9 & \bulhyperlink{UC7}{UC7}\\ \hline
    RDF10 & \bulhyperlink{UC8}{UC8}\\ \hline
    RDF11 & \bulhyperlink{UC8}{UC8}, \bulhyperlink{UC8.1}{UC8.1}\\ \hline
    RDF12 & \bulhyperlink{UC9}{UC9}\\ \hline
    ROF13 & \bulhyperlink{UC10}{UC10}\\ \hline
    ROF14 & \bulhyperlink{UC10}{UC10}, Capitolato\\ \hline
    ROF15 & Verbale interno\\ \hline
    ROF16 & Verbale interno\\ \hline
    RZF17 & \bulhyperlink{UC11}{UC11}, \bulhyperlink{UC11.1}{UC11.1}\\ \hline
    RZF18 & \bulhyperlink{UC12}{UC12}, \bulhyperlink{UC12.1}{UC12.1}\\ \hline
    RZF19 & \bulhyperlink{UC13}{UC13}\\ \hline
    RZF20 & \bulhyperlink{UC15}{UC15}\\ \hline
    RZF21 & \bulhyperlink{UC15}{UC15}, \bulhyperlink{UC16}{UC16}, \bulhyperlink{UC17}{UC17}\\ \hline
    ROQ1 & Verbale Interno\\ \hline
    ROQ2 & Verbale Interno\\ \hline
    ROQ3 & Capitolato \\ \hline
    \end{tabularx}
    \caption{Tracciamento Requisito - Fonti}
\end{table}

\newpage
\begin{table}[h!]
    \centering
    \renewcommand{\arraystretch}{1.6} % Per aumentare l'altezza delle righe
    \begin{tabularx}{0.8\textwidth}{|>{\centering\arraybackslash}p{2.8cm}|>{\centering\arraybackslash}X|} \hline
    \rowcolor[HTML]{FFD700} 
    \textbf{Requisito} & \textbf{Fonti} \\ \hline
    ROQ4 & Capitolato \\ \hline
    ROV1 & Verbale esterno\\ \hline
    ROV2 & Capitolato\\ \hline
    ROV3 & Capitolato\\ \hline
    ROV4 & Capitolato\\ \hline
    ROI1 & Verbale interno\\ \hline
    ROI2 & Verbale interno\\ \hline
    ROI3 & Verbale interno\\ \hline
    ROI4 & Verbale esterno\\ \hline
    \end{tabularx}
    \caption{Tracciamento Requisito - Fonti}
    \label{tab:Tracciamento_requisiti_fonti}
\end{table}