% Intestazione
\fancyhead[L]{3 \hspace{0.2cm} Diario della riunione} % Testo a sinistra


\section{Diario della riunione}

\begin{itemize}
    \item Presentazione generale dell'azienda \emph{proponente}\textsubscript{\textit{\textbf{G}}} e presentazione del piano di lavoro e della modalità dei rapporti: 
    \emph{AzzurroDigitale} propone incontri periodici via \emph{Discord}\textsubscript{\textit{\textbf{G}}} per pianificare e discutere lo stato di avanzamento del 
    progetto e per chiarire eventuali dubbi. Questi incontri si terranno inizialmente ogni 2 settimane, mentre da gennaio in poi si terranno ogni settimana, sempre di giovedì.
    \item È stato proposto da \emph{AzzurroDigitale} il modello di sviluppo \emph{Agile}\textsubscript{\textit{\textbf{G}}} per il progetto didattico, che prevede che in una 
    \emph{Sprint}\textsubscript{\textit{\textbf{G}}} si debbano seguire le seguenti fasi:
    \begin{itemize}
        \item Raccolta specifiche
        \item Creazione Ticket
        \item Planning Poker
        \item Sviluppo e avanzamento ticket
        \item Review
    \end{itemize}
    \item Proposta di \emph{AzzurroDigitale} di utilizzo di due strumenti a supporto della metodologia Agile: \emph{Confluence}\textsubscript{\textit{\textbf{G}}} e 
    \emph{Jira}\textsubscript{\textit{\textbf{G}}}. Inoltre, il contenuto creato per il progetto al loro interno (documentazione e specifiche su \emph{Confluence} 
    e ticket su \emph{Jira}) potrà essere utilizzato come base per l’allenamento del chatbot.
    \item Definiti i prossimi immediati passi da compiere:
    \begin{itemize}
        \item Comunicazione username Discord
        \item Creazione gruppi Discord
        \item Sviluppo Piano di Lavoro Generale e eventuale Budget
        \item Condivisione obiettivi delle prime 3/4 Sprint
        \item Prima Raccolta Specifiche
    \end{itemize}
    \item Chiarimento di vari ulteriori dubbi emersi nella riunione, che saranno ulteriomente discussi nei prossimi incontri esterni. Alcuni di questi sono:
    \begin{itemize}
        \item Il \emph{Proof of Concept}\textsubscript{\textit{\textbf{G}}} deve essere composto da poche \emph{feature}\textsubscript{\textit{\textbf{G}}} 
        ma funzionanti.
        \item Le infomazioni che il bot dovrà estrarre da \emph{GitHub}\textsubscript{\textit{\textbf{G}}} devono riguardare:
        \begin{itemize}
            \item il singolo \emph{snippet}\textsubscript{\textit{\textbf{G}}} di codice,
            \item la data del corrispondente commit
            \item l'autore del commit.
        \end{itemize}
        \item È importante fare un'analisi della sicurezza dei dati immessi e gestiti dal bot, per ragioni di privacy, e quest'analisi la dobbiamo svolgere fin da
        subito, in parallelo con lo studio delle tecnologie.
        \item La documentazione richiesta da \emph{AzzurroDigitale} a progetto terminato sarà:
        \begin{itemize}
            \item Analisi funzionale e tecnica
            \item Dettaglio dei test sviluppati
            \item Verbale del collaudo, cioè il verbale dell'incontro in cui verrà svolta la \emph{demo}\textsubscript{\textit{\textbf{G}}} finale del prodotto.
        \end{itemize}
    \end{itemize}
\end{itemize}