% Intestazione
\fancyhead[L]{C \hspace{0.2cm} Metriche per la qualità} % Testo a sinistra

\pagenumbering{roman} % Fa ripartire la numerazione romana delle pagine da I


\section{Metriche per la qualità}
\label{sec:metriche_qualita}

La valutazione della qualità del software è un \emph{processo}\textsubscript{\textit{\textbf{G}}} complesso che coinvolge diverse
categorie di \emph{metriche}\textsubscript{\textit{\textbf{G}}}. Esse sono suddivise principalmente in quattro macro-aree: metriche
interne, metriche esterne, metriche della qualità in uso e metriche per la qualità di processo.
Ogni categoria ha il suo scopo specifico nel valutare aspetti diversi del \emph{ciclo di vita del
software}\textsubscript{\textit{\textbf{G}}}.


\subsection{Metriche interne}
Le metriche interne sono utilizzate per valutare il codice sorgente e la documentazione
intermedia durante lo sviluppo del software. Esse forniscono indicazioni sulla qualità interna
del prodotto e possono aiutare a prevenire potenziali problemi futuri. Queste metriche
sono spesso utilizzate durante il processo di sviluppo per migliorare la qualità del codice e
facilitare la manutenzione.


\subsection{Metriche esterne}
Le metriche esterne sono orientate verso gli utenti finali e valutano il comportamento del
software in esecuzione. Esse misurano le caratteristiche che sono visibili agli utenti, come le
prestazioni e la facilità d’uso. Queste metriche sono essenziali per garantire che il software
soddisfi le aspettative degli utenti e fornisca un’esperienza positiva.


\subsection{Metriche della qualità in uso}
Le metriche della qualità in uso valutano come gli utenti effettivamente interagiscono e
sfruttano il software nell’ambiente operativo. Misurano l’\emph{efficacia}\textsubscript{\textit{\textbf{G}}}, 
l’\emph{efficienza}\textsubscript{\textit{\textbf{G}}}, la soddisfazione dell’utente e altri aspetti che 
influenzano direttamente l’esperienza dell’utente durante l’utilizzo del software.


\subsection{Metriche per la qualità di processo}
Le metriche per la qualità di processo valutano la qualità dei processi adottati durante 
lo sviluppo del software. Esse includono metriche per miglioramento, fornitura, 
\emph{codifica}\textsubscript{\textit{\textbf{G}}} e documentazione, fornendo indicazioni su come i processi 
possono essere ottimizzati per migliorare la qualità del prodotto finale:
\begin{itemize}
    \item \textbf{Miglioramento}: queste metriche valutano l’efficacia dei processi di miglioramento continuo implementati
    nel ciclo di vita dello sviluppo del software.
    \item \textbf{Fornitura}: le metriche di fornitura misurano la qualità del processo di consegna del software, compreso 
    il rispetto dei tempi, la gestione delle risorse e la conformità agli standard.
    \item \textbf{Codifica}: queste metriche valutano la qualità del processo di scrittura del codice, inclusa la correttezza 
    sintattica, la chiarezza e la conformità agli standard di codifica.
    \item \textbf{Documentazione}: misurano la qualità della documentazione associata al software, fornendo indicazioni sulla
    chiarezza e completezza della documentazione.
\end{itemize}


\subsection{Metriche di qualità di prodotto}
Le metriche per la qualità di prodotto valutano le caratteristiche del software come funzionalità, 
usabilità, manutenibilità e altre. Queste metriche forniscono indicazioni specifiche
sulla qualità del prodotto software in termini di conformità agli standard e soddisfazione
degli utenti.
\begin{itemize}
    \item \textbf{Funzionalità}: queste metriche valutano la completezza e la correttezza delle funzionalità del software,
    assicurando che risponda adeguatamente ai requisiti
    \item \textbf{Usabilità}: misurano la facilità con cui gli utenti possono interagire con il software, considerando aspetti
    come la comprensibilità, l’apprendibilità e l’operabilità.
\end{itemize}


\subsection{Manutenibilità}
La manutenibilità è una categoria specifica che riflette la facilità con cui il software può
essere modificato, corretto e adattato nel tempo. Essa include metricheG di affidabilità,
valutando la capacità del software di mantenere prestazioni stabili e coerenti anche dopo le
modifiche.
\begin{itemize}
    \item \textbf{Affidabilità}: Questa metrica valuta la capacità del software di mantenere la coerenza delle prestazioni anche 
    dopo l’introduzione di modifiche, assicurando che nuove funzionalità non compromettano l’integrità del sistema.
\end{itemize}