% Intestazione
\fancyhead[L]{4 \hspace{0.2cm} Processi organizzativi} % Testo a sinistra


\section{Processi organizzativi}
\label{sec:processi_organizzativi}

\subsection{Scopo}
Lo scopo di questa sezione è esporre le modalità e gli strumenti di coordinamento usati dal gruppo per la comunicazione interna ed esterna, 
definire e normare l’assegnazione di ruoli e compiti, e stabilire procedure e metodologie che il team seguirà durante il 
ciclo di vita del progetto.

\subsection{Aspettative}
Le aspettative in questa fase sono le seguenti:
\begin{itemize}
    \item Redazione del documento \emph{Piano di Progetto}\textsubscript{\textit{\textbf{G}}};
    \item Definire i ruoli assunti dai membri del gruppo;
    \item Definizione di un piano per l'esecuzione dei compiti programmati.
\end{itemize}

\subsection{Descrizione}
Le attività previste da tale processo sono raccolte nel \emph{Piano di Progetto}.
In dettaglio viene trattata la gestione dei seguenti argomenti:
\begin{itemize}
    \item Ruoli;
    \item Incontri;
    \item Comunicazioni;
    \item Metodo di sviluppo e gestione delle attività;
    \item Miglioramento
    \item Formazione;
\end{itemize}

\subsection{Gestione dei Processi}
\subsubsection{Gestione dei ruoli}
Il \emph{Responsabile di Progetto} ha il compito di suddividere i ruoli e l’assegnazione oraria per i membri del gruppo, 
garantendo che ognuno di essi assuma, nel corso del progetto, almeno una volta ogni ruolo. 
Visto che nelle varie fasi di sviluppo del progetto le attività da svolgere variano, non sempre sarà necessario coprire tutti i ruoli.
I ruoli richiesti dal progetto sono descritti qui di seguito. 

\subsubsubsection{Responsabile di Progetto}
È il punto di riferimento per tutto il gruppo e per le comunicazioni con il \emph{committente} e l'\emph{azienda proponente}. 
Si assume la responsabilità delle scelte del gruppo dopo averle approvate e coordina le azioni dei vari membri del team. 
È responsabile della pianificazione, dell’esecuzione e del monitoraggio delle attività del progetto secondo le specifiche previste dal gruppo.
Le sue principali responsabilità includono:
\begin{itemize}
    \item \textbf{Pianificazione e Definizione degli Obiettivi}: pianificare e definire obiettivi, scadenze e risorse necessarie per il progetto;
    \item \textbf{Coordinamento del Team e Gestione delle Risorse Umane}: coordinare il lavoro del team, gestire le risorse umane, assegnare i compiti agli altri membri del gruppo e garantire che ogni membro del gruppo sappia cosa ci si aspetta da lui;
    \item \textbf{Monitoraggio del Progresso e Gestione dei Rischi}: monitorare il progresso del progetto, assicurandosi che le attività pianificate vengano svolte entro i tempi e le modalità previste, e identificare, studiare e gestire i rischi associati al progetto;
    \item \textbf{Comunicazione}: comunicare con il team, le parti interessate, il \emph{committente} e l’\emph{azienda proponente}. Gestire le comunicazioni e gli incontri interni, nonché le relazioni esterne;
    \item \textbf{Approvazione della Documentazione}: approvare la documentazione prodotta dal gruppo, l’offerta economica sottoposta al committente e qualsiasi \emph{task} completata e verificata;
    \item \textbf{Suddivisione delle Attività}: suddividere le attività del gruppo in singole \emph{issue}\textsubscript{\textit{\textbf{G}}}, e gestire la loro assegnazione ai membri del gruppo.
\end{itemize}

\subsubsubsection{Amministratore}
È incaricato del controllo e dell’amministrazione di tutto l’ambiente di lavoro. Le sue principali responsabilità includono:
\begin{itemize}
    \item \textbf{Gestione della Documentazione}: salvaguardare la documentazione di progetto e gestire il sistema di archiviazione e versionamento di documentazione e codice;
    \item \textbf{Gestione degli Strumenti e dell'Infrastruttura}: gestire l’infrastruttura e gli strumenti utilizzati, mantenere efficiente l’ambiente di sviluppo e fornire strumenti adeguati ai membri del gruppo. Gestire errori e segnalazioni di malfunzionamenti con gli strumenti tecnologici;
    \item \textbf{Automazione e Miglioramento dei Processi}: automatizzare i processi, individuare punti di miglioramento nei processi e mettere in opera risorse per migliorare l’ambiente di lavoro;
    \item \textbf{Gestione della Configurazione e Versionamento}: effettuare il controllo di versioni e configurazioni del prodotto software, gestire il sistema di configurazione e \emph{versionamento}\textsubscript{\textit{\textbf{G}}} del prodotto;
    \item \textbf{Gestione della Qualità}: redigere e attuare i piani e le procedure per la gestione della qualità, garantendo l’\emph{efficacia}\textsubscript{\textit{\textbf{G}}} e l’\emph{efficienza}\textsubscript{\textit{\textbf{G}}} dei processi;
    \item \textbf{Supporto e Comunicazione}: gestire le comunicazioni interne, supportare la gestione delle risorse e delle comunicazioni del progetto.
\end{itemize}

\subsubsubsection{Analista}
L’analista è la figura responsabile dell’analisi dei requisiti del progetto e della definizione delle specifiche. 
Le sue principali responsabilità includono:
\begin{itemize}
    \item \textbf{Raccolta e Analisi dei Requisiti}: raccogliere e analizzare i requisiti del cliente, trasformando i bisogni del \emph{proponente} nelle aspettative che il gruppo deve soddisfare per sviluppare un prodotto professionale;
    \item \textbf{Definizione delle Specifiche}: definire le \emph{specifiche funzionali}\textsubscript{\textit{\textbf{G}}} e \emph{tecniche}\textsubscript{\textit{\textbf{G}}} del prodotto, modellare concettualmente il sistema e suddividere i requisiti del progetto in categorie;
    \item \textbf{Collaborazione con il Progettista}: collaborare con il progettista per creare soluzioni che soddisfino i requisiti definiti;
    \item \textbf{Documentazione}: redigere l'\emph{Analisi dei Requisiti}\textsubscript{\textit{\textbf{G}}}, documentando i servizi che il sistema deve fornire e garantendo che i requisiti siano chiari e completi;
    \item \textbf{Studio del Problema e del Contesto Applicativo\textsubscript{\textit{\textbf{G}}}}: studiare il problema e il relativo \emph{contesto applicativo}, comprendere la complessità del problema e definire i requisiti impliciti ed espliciti;
    \item \textbf{Gestione della Qualità}: assicurare che i requisiti siano ben definiti per evitare problemi durante la fase di progettazione e sviluppo.
\end{itemize}

\subsubsubsection{Progettista}
È responsabile di modellare i requisiti individuati nella fase di analisi e ricomporli in un’\emph{architettura}\textsubscript{\textit{\textbf{G}}} che possa soddisfarli. 
Le sue principali responsabilità includono:
\begin{itemize}
    \item \textbf{Definizione dell'Architettura}: creare un'\emph{architettura} del sistema che soddisfi i requisiti richiesti, con un alto livello di manutenibilità e un basso grado di \emph{accoppiamento}\textsubscript{\textit{\textbf{G}}} tra i componenti;
    \item \textbf{Scelte Tecniche e Tecnologiche}: effettuare scelte riguardanti gli aspetti tecnici e tecnologici del progetto, favorendo l'\emph{efficacia} e l'\emph{efficienza}. Scegliere eventuali \emph{pattern architetturali}\textsubscript{\textit{\textbf{G}}} da implementare e sviluppare lo \emph{schema UML}\textsubscript{\textit{\textbf{G}}} delle classi;
    \item \textbf{Qualità del Prodotto}: garantire la qualità del prodotto, producendo una soluzione economica e mantenibile che rientri nei costi stabiliti nel preventivo;
    \item \textbf{Collaborazione e Coordinamento}: collaborare con gli analisti per comprendere i requisiti e con i programmatori per l'implementazione delle soluzioni tecniche. Coordinare le attività di progettazione per assicurare che il prodotto finale soddisfi tutti i requisiti;
    \item \textbf{Ricerca e Innovazione}: approfondire le conoscenze tecniche e ricercare strumenti tecnologici utili nell'ambito di applicazione, migliorando continuamente l'\emph{architettura} del sistema.
\end{itemize}

\subsubsubsection{Programmatore}
Il programmatore è responsabile della scrittura del codice e dell’implementazione delle soluzioni tecniche definite dal progettista. 
Le sue principali responsabilità includono:
\begin{itemize}
    \item \textbf{Scrittura del Codice}: scrivere codice che soddisfi le specifiche di progettazione, applicando le "\emph{best practices}\textsubscript{\textit{\textbf{G}}}" note riguardanti la scrittura di codice;
    \item \textbf{Risoluzione dei Problemi Tecnici}: risolvere i problemi tecnici che emergono durante lo sviluppo;
    \item \textbf{Test del Codice}: testare il proprio codice per garantire che funzioni correttamente e soddisfi i requisiti;
    \item \textbf{Manutenibilità del Codice}: scrivere codice documentato, mantenibile e versionato, rendendo l'attività di verifica semplice e agevole;
    \item \textbf{Documentazione}: redigere le documentazioni necessarie per la comprensione e l'uso del codice;
    \item \textbf{Collaborazione}: collaborare con il progettista e altri membri del team per implementare l'architettura prodotta nella fase di progettazione.
\end{itemize}

\subsubsubsection{Verificatore}
Il verificatore è responsabile del controllo del lavoro svolto dagli altri componenti del gruppo, assicurando che sia conforme agli standard di qualità e alle norme di progetto. 
Le sue principali responsabilità includono:
\begin{itemize}
    \item \textbf{Controllo della Conformità}: verificare che ogni file caricato in un \emph{branch}\textsubscript{\textit{\textbf{G}}} protetto della \emph{repository}\textsubscript{\textit{\textbf{G}}} sia conforme alle \emph{Norme di Progetto}\textsubscript{\textit{\textbf{G}}}, controllando i file modificati o aggiunti durante una \emph{pull request}\textsubscript{\textit{\textbf{G}}} per errori ortografici, sintattici, logici e di \emph{build}\textsubscript{\textit{\textbf{G}}};
    \item \textbf{Esame dei Prodotti}: esaminare i prodotti in fase di revisione utilizzando le tecniche e gli strumenti definiti nelle \emph{Norme di Progetto}, assicurandosi che siano conformi ai requisiti funzionali e di qualità;
    \item \textbf{Segnalazione degli Errori}: segnalare eventuali errori riscontrati durante la \emph{verifica}, fornendo \emph{feedback}\textsubscript{\textit{\textbf{G}}} completi e chiari a chi ha prodotto il lavoro revisionato;
    \item \textbf{Assicurazione della Qualità}: assicurare che la qualità di quanto prodotto sia conforme agli standard imposti, verificando la conformità di ogni stadio del \emph{ciclo di vita}\textsubscript{\textit{\textbf{G}}} del prodotto;
    \item \textbf{Documentazione}: redigere la parte retrospettiva del \emph{Piano di Qualifica}\textsubscript{\textit{\textbf{G}}}, descrivendo le verifiche e le prove effettuate;
    \item \textbf{Sorveglianza Continua}: essere presente durante l’intera durata del progetto, garantendo che le attività svolte rispettino il livello di qualità atteso.
\end{itemize}

\subsubsection{Gestione degli Incontri}
Le riunioni si dividono in interne ed esterne. Per ogni riunione verrà redatto un verbale per tenere traccia degli argomenti trattati e delle decisioni prese. 
La pratica di redigere verbali favorisce la trasparenza e la tracciabilità delle decisioni, costituendo uno strumento prezioso per mantenere un registro chiaro e condiviso delle attività svolte.

\subsubsubsection{Riunioni Interne}
Le riunioni interne si svolgono esclusivamente tra i membri del gruppo e avvengono almeno una volta a settimana. 
Un incontro può essere richiesto da qualsiasi membro del gruppo al \emph{Responsabile di Progetto}, che lo organizzerà in base alla disponibilità di tutti i membri. 
La modalità di incontro è prevalentemente virtuale, utilizzando come piattaforma di riferimento \emph{Discord}\textsubscript{\textit{\textbf{G}}}.
Per mantenere una buona efficienza e ottimizzare il tempo, ogni riunione interna segue una linea guida:
\begin{itemize}
    \item \textbf{Preparazione}: Prima di ogni incontro, viene preparata una scaletta dei principali punti da discutere;
    \item \textbf{Discussione}: Discussione del lavoro svolto da ogni membro del gruppo dall'ultimo incontro e dei punti prefissati nella scaletta iniziale, con confronto su eventuali dubbi;
    \item \textbf{Pianificazione}: Pianificazione delle attività da svolgere per ogni membro del gruppo fino al prossimo incontro.
\end{itemize}
Alla fine di ogni incontro viene scelto un membro del gruppo incaricato di redigere il verbale interno con una breve descrizione dei punti focali dell'incontro.

\subsubsubsection{Riunioni Esterne}
Le riunioni esterne coinvolgono i membri del gruppo e i referenti aziendali. Questi incontri sono previsti di giovedì pomeriggio (dalle 17.00 alle 18.00) con cadenza bisettimanale nella prima parte di sviluppo del progetto, 
indicativamente fino al superamento della revisione \emph{RTB}\textsubscript{\textit{\textbf{G}}}, per poi passare ad una cadenza settimanale fino alla consegna definitiva del progetto. 
È possibile per i rappresentanti di entrambe le parti richiedere incontri aggiuntivi se necessario.
Le riunioni avvengono principalmente in modalità virtuale, utilizzando come piattaforma di riferimento \emph{Google Meet}\textsubscript{\textit{\textbf{G}}}.
Alla fine di ogni incontro esterno, viene redatto un verbale per documentare i momenti salienti e le decisioni prese e viene sottoposto ai referenti aziendali per approvazione.

\subsubsection{Gestione delle Comunicazioni}
Le comunicazioni per tutta la durata del progetto si dividono in interne ed esterne. Ogni canale di comunicazione è dedicato a specifiche esigenze per garantire efficienza e chiarezza.

\subsubsubsection{Comunicazioni Interne}
Le comunicazioni interne riguardano esclusivamente i membri del gruppo e avvengono attraverso i seguenti strumenti:
\begin{itemize}
    \item \textbf{\emph{Telegram}}\textsubscript{\textit{\textbf{G}}}: Utilizzato per comunicazioni rapide e istantanee, sia testuali che vocali, e per la pianificazione degli incontri interni;
    \item \textbf{Discord}: Utilizzato per le riunioni interne e per comunicazioni veloci tra i membri del gruppo. \emph{Discord} permette una comunicazione sincrona vocale affidabile e comoda per tutti i membri del gruppo.
\end{itemize}

\subsubsubsection{Comunicazioni Esterne}
Le comunicazioni esterne sono gestite dal \emph{Responsabile di Progetto} e coinvolgono il proponente e i committenti. Avvengono attraverso i seguenti canali:
\begin{itemize}
    \item \textbf{Google Meet}: Utilizzato per le riunioni esterne con il proponente e i committenti. Queste riunioni possono essere richieste da entrambe le parti e al termine viene redatto un verbale;
    \item \textbf{Discord}: Utilizzato per comunicazioni veloci con i referenti aziendali;
    \item \textbf{Email}: Utilizzata per comunicazioni formali e dettagliate. L'indirizzo email del gruppo è accessibile a tutti i membri del team per garantire una comunicazione chiara e trasparente.
\end{itemize}

\subsubsubsection{Compiti del Responsabile del Progetto}
Il \emph{Responsabile di Progetto} si occupa di:
\begin{itemize}
    \item Pianificare l'ordine del giorno delle riunioni;
    \item Comunicare tempestivamente eventuali variazioni orarie;
    \item Verificare la presenza dei membri durante le riunioni;
    \item Guidare le discussioni in modo ordinato;
    \item Nominare un segretario per redigere il verbale;
    \item Approvare formalmente il verbale.
\end{itemize}

\subsubsubsection{Doveri dei partecipanti}
I Partecipanti si impegnano a:
\begin{itemize}
    \item Partecipare puntualmente alle riunioni;
    \item Comunicare tempestivamente eventuali ritardi o assenze;
    \item Partecipare attivamente durante le riunioni;
    \item Mantenere un comportamento consono durante le riunioni.
\end{itemize}

\subsubsection{Gestione compiti e task}
La gestione dei compiti e task è cruciale per garantire un’organizzazione efficace delle attività, mantenendo un alto livello di tracciabilità, qualità e aderenza alle tempistiche pianificate. 
Questa sezione descrive le metodologie adottate dal team e il ciclo di vita di ciascun task, che ne scandisce lo svolgimento dall’ideazione alla chiusura.

\subsubsubsection{Metodologie e pratiche}
Il team utilizza il metodo di sviluppo \emph{Agile}\textsubscript{\textit{\textbf{G}}} per organizzare e pianificare le attività progettuali. 
Questo approccio consente una gestione flessibile e iterativa, promuovendo una costante collaborazione tra i membri del team e una comunicazione trasparente con il committente. 
Le principali pratiche includono:
\begin{itemize}
    \item \textbf{Organizzazione attraverso Sprint\textsubscript{\textit{\textbf{G}}}}: ogni ciclo di lavoro è scandito da sprint di durata predefinita, con obiettivi chiari e misurabili;
    \item \textbf{Collaborazione strutturata}: i task vengono definiti, pianificati e assegnati utilizzando \emph{Jira}\textsubscript{\textit{\textbf{G}}}, cioè uno strumento di \emph{Issue Tracking System}\textsubscript{\textit{\textbf{G}}} che facilita la condivisione delle responsabilità e il lavoro di gruppo;
    \item \textbf{Monitoraggio continuo}: board visive, roadmap e \emph{diagrammi di Gantt}\textsubscript{\textit{\textbf{G}}} offrono una visione d'insieme sullo stato delle attività, permettendo di identificare rapidamente eventuali blocchi o ritardi;
    \item \textbf{Revisione e miglioramento iterativo}: al termine di ogni sprint, il team valuta i risultati ottenuti e pianifica miglioramenti per il ciclo successivo.
\end{itemize}
Questo approccio garantisce flessibilità nell'adattarsi a cambiamenti nei requisiti e favorisce il coinvolgimento attivo di tutti i partecipanti.

\subsubsubsection{Ciclo di vita di un Task}
Il \emph{ciclo di vita}\textsubscript{\textit{\textbf{G}}} dei task si articola in sei fasi principali, gestite attraverso \emph{Jira}, che funge da strumento digitale configurato per garantire efficienza e coerenza.
\begin{itemize}
    \item \textbf{Creazione}: il \emph{Responsabile} definisce i task su \emph{Jira}, completando i campi essenziali (titolo, descrizione, priorità, assegnatari, stima temporale) e assegnandoli ad una sprint; checklist e tag possono essere aggiunti per migliorare l’organizzazione;
    \item \textbf{Assegnazione}: i task vengono distribuiti ai membri in base alle competenze e al carico di lavoro. Nei casi in cui un task rimanga non assegnato, i membri possono prendersene carico autonomamente;
    \item \textbf{Esecuzione}: l’esecutore del task ne aggiorna lo stato (es. da "To Do" a "In Progress") e lavora su un branch dedicato per garantire una gestione separata e tracciabile delle modifiche;
    \item \textbf{Revisione}: una volta completato, il task passa in revisione. Il \emph{Verificatore} esamina il lavoro, segnala eventuali modifiche necessarie tramite commenti o review e valida il task se conforme agli standard;
    \item \textbf{Accettazione}: dopo la revisione, il \emph{Responsabile} approva il task, esegue il merge del \emph{branch} (se pertinente) e aggiorna lo stato del task a "Completato".
\end{itemize}    
Questa gestione strutturata dei task consente al team di affrontare le attività con precisione, tracciando ogni modifica e garantendo una visione chiara e condivisa dell’avanzamento del progetto.

\subsubsection{Miglioramento}
\textbf{Scopo}: durante lo svolgimento delle attività e la successiva elaborazione della documentazione, ci poniamo l'obiettivo di seguire costantemente il principio di miglioramento continuo. 
L'obiettivo principale è identificare proattivamente le attività, i ruoli e le opportunità di miglioramento, cercando nuove soluzioni per affrontare sfide emergenti o passate.\\ \\
\textbf{Descrizione}: il miglioramento continuo è un ciclo iterativo che ci consente di adattarci dinamicamente alle esigenze mutevoli del progetto, 
garantendo una crescita continua e una maggiore efficienza nelle nostre attività.
Questo approccio contraddistinguerà ogni fase del processo, dalla pianificazione all'esecuzione, compresa la documentazione.\\ \\
\textbf{Implementazione}: durante lo svolgimento delle attività e la stesura dei documenti, il team si impegna a seguire il principio di miglioramento continuo. 
Le problematiche riscontrate e le soluzioni adottate vengono documentate e confrontate per garantire una gestione consapevole e mirata delle sfide incontrate durante il processo di sviluppo. 
Questo approccio formale è stato adottato per evitare di ripetere errori già fatti e fornire soluzioni efficaci.

\subsubsection{Formazione}
\textbf{Scopo}: l'obiettivo di questa sezione è stabilire le regole relative al processo di istruzione dei membri del team, 
che include lo studio delle tecnologie impiegate per la produzione dei documenti e la realizzazione del prodotto richiesto.
\textbf{Aspettative}: le aspettative per il processo di istruzione includono:
\begin{itemize}
    \item Acquisire una solida conoscenza del linguaggio \emph{\LaTeX}\textsubscript{\textit{\textbf{G}}};
    \item Ottenere una buona dimestichezza con i diversi linguaggi di programmazione, le librerie e gli strumenti necessari per la realizzazione del prodotto software assegnato dal proponente;
    \item Sviluppare una buona familiarità con l'ambiente in cui si sta lavorando relativamente al capitolato di interesse.
\end{itemize}
\textbf{Piano di Formazione}: ogni componente del gruppo è libero di formarsi nel modo che ritiene più opportuno, sostenendo il processo di apprendimento "learning by doing". Questa libertà permette di ampliare la visione del gruppo riguardo le tecnologie utilizzate, aumentando la consapevolezza e migliorando complessivamente le scelte implementative.
