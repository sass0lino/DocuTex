\appendix % Cambia la numerazione delle sezioni da numeri a lettere


% Intestazione
\fancyhead[L]{A} % Testo a sinistra

\section{}
%\addcontentsline{toc}{section}{A}

\hypertarget{sec:accoppiamento}{}
\subsection*{Accoppiamento}
In informatica, l'accoppiamento è il grado di dipendenza tra due o più componenti di un sistema software. Un accoppiamento elevato indica una forte
interdipendenza tra le componenti, mentre un accoppiamento basso indica una minore dipendenza. Un basso accoppiamento è generalmente preferibile, poiché
rende il sistema più flessibile, modulare e facile da mantenere.

\hypertarget{sec:accessibilità}{}
\subsection*{Accessibilità}
Capacità di rendere le informazioni sono più facilmente fruibili, condivisibili e adattabili alle esigenze di ciascun utente.

\hypertarget{sec:modello_agile}{}
\subsection*{Agile, modello di sviluppo}
Approccio alla gestione dei progetti e allo sviluppo del software che enfatizza la flessibilità, la collaborazione e il miglioramento continuo. 
Ha come principi fondamentali consegna continua, collaborazione stretta tra sviluppatori e clienti, adattabilità ai cambiamenti, 
comunicazione diretta tra le parti e valutazione e miglioramento continuo del processo e del prodotto.

\subsection*{Analisi dei Requisiti}
Processo fondamentale dello sviluppo di un prodotto software che si concentra sulla raccolta, analisi e definizione delle necessità e delle aspettative 
degli utenti finali, degli stakeholder e del sistema nel suo complesso. Questo processo mira a comprendere e documentare in modo chiaro e completo le 
esigenze, le funzionalità, le prestazioni e i vincoli che il sistema deve soddisfare. L’obiettivo principale dell’analisi dei requisiti è fornire una 
base solida per tutte le fasi successive dello sviluppo del software, assicurando che il prodotto finale soddisfi le esigenze degli utenti e raggiunga 
gli obiettivi del progetto.

\subsection*{Analisi dei Rischi}
L'analisi dei rischi è il processo di identificazione, valutazione e priorizzazione dei rischi in un progetto, sistema o attività, al fine di ridurre 
o gestire il loro impatto potenziale. Viene utilizzata per prevedere gli eventi negativi che potrebbero influenzare il successo di un progetto e per 
determinare le azioni preventive o correttive da intraprendere.

\hypertarget{sec:analisi_dinamica}{}
\subsection*{Analisi dinamica}
Esecuzione del software per verificarne il comportamento, identificare malfunzionamenti e valutare il rispetto dei requisiti in un ambiente reale o simulato.

\hypertarget{sec:analisi_statica}{}
\subsection*{Analisi statica}
Controlli eseguiti senza avviare il software, utilizzati per verificare la qualità del codice, individuare errori e garantire conformità agli standard.

\hypertarget{sec:analogico}{}
\subsection*{Analogico}


\hypertarget{sec:angular}{}
\subsection*{Angular}
Framework open-source sviluppato da Google per creare applicazioni web dinamiche e scalabili. Basato su TypeScript, utilizza un'architettura a componenti, 
il data binding bidirezionale e strumenti integrati per lo sviluppo di interfacce utente robuste e performanti.

\hypertarget{sec:api}{}
\subsection*{API}
Acronimo di Application Programming Interface, è un insieme di regole e protocolli che consente a diverse applicazioni software di comunicare tra loro 
per scambiare dati, caratteristiche e funzionalità. Le API fungono da intermediari, permettendo lo scambio di informazioni tra software diversi, semplificando 
e accelerando lo sviluppo di applicazioni.

\hypertarget{sec:applicazione_web}{}
\subsection*{Applicazione web}
Applicazione accessibile via web tramite un browser (Safari, Chrome, Firefox, Edge etc.) e può funzionare sia da mobile che da desktop. Si tratta di un'architettura 
tipicamente di tipo client-server, che offre determinati servizi all'utente.

\hypertarget{sec:architettura}{}
\subsection*{Architettura}
In ambito informatico, l'organizzazione di base di un sistema, espressa dai suoi componenti, dalle relazioni tra di loro e con l'ambiente, 
e dai principi che ne guidano il progetto e l'evoluzione. Essa comprende le strutture del sistema, necessarie per ragionare su di esso, 
che includono elementi software, le relazioni tra di essi e le loro proprietà.

\hypertarget{sec:attore}{}
\subsection*{Attore}
Nel contesto dell'analisi dei requisiti e del design del software, il termine attore rappresenta chiunque interagisca con il sistema. Gli attori possono 
essere utenti, altri sistemi o qualsiasi altra entità che abbia un ruolo nelle interazioni con il sistema modellato.

\subsection*{AzzurroDigitale}
AzzurroDigitale è una società italiana con sede a Padova. Si occupa di digitalizzazione di processi sia con prodotti proprietari che di terze parti, e 
ha l’obiettivo di accompagnare le aziende manifatturiere nella transizione 5.0.

\newpage


% Intestazione
\fancyhead[L]{B} % Testo a sinistra

\section{}

\hypertarget{sec:back-end}{}
\subsection*{Back-end}
Termine che si riferisce alla parte di un'applicazione web o di un sistema software che gestisce le funzioni e le 
elaborazioni necessarie per rendere possibile l'utilizzo dell'applicazione stessa. Esso comprende tutti i componenti, 
i server e i sistemi di archiviazione dati che non sono accessibili direttamente dall'utente finale, ma che lavorano 
dietro le quinte per fornire funzionalità e dati al frontend. Normali attività gestite dal back-end includono la 
gestione dei database, l'elaborazione delle richieste dei client, la logica di business e la sicurezza 
dell'applicazione.

\subsection*{Backlog}
Insieme di compiti/attività da completare per un certo obiettivo. All’interno del framework Scrum, ne esistono due tipi principali: il product backlog, 
che è la lista delle funzionalità da implementare, e lo sprint backlog, che contiene le attività da svolgere durante un particolare sprint.
Un’attività interna al backlog porta valore ad un progetto perchè possiede:
\begin{itemize}
    \item Stato, che segnala se l’attività è stata completata, in corso o non ancora iniziata.
    \item Priorità, che indica l’importanza dell’attività rispetto alle altre.
    \item Assegnatario, cioè una persona incaricata a svolgere l’attività. Questa assegnazione non è vincolante, infatti se un membro del team ha terminato 
    la sua attività può prendersi a carico un’altra attività presente nel backlog anche se non era stata inizialmente assegnata a lui.
    \item Scadenza, cioè un termine temporale entro il quale l’attività deve essere svolta.
\end{itemize}

\hypertarget{sec:baseline}{}
\subsection*{Baseline}
Nel contesto dell'ingegneria del software, stato di avanzamento che rappresenta un insieme di punti di arrivo che ci si pone come obiettivo di raggiungere 
in una milestone, dimostrando che l'incremento delle modifiche condotte ha portato a un risultato.

\subsection*{Best Practices}
Nello sviluppo software, metodologie che attraverso l’esperienza e la sperimentazione sono state identificate come modi efficaci e raccomandati di 
affrontare determinati problemi o compiti nel processo di sviluppo del software. Queste pratiche sono considerate migliori (best) perché hanno dimostrato 
di portare a risultati di alta qualità, facilitando la manutenzione del codice e promuovendo una migliore collaborazione nel team di sviluppo.

\hypertarget{sec:Bot}{}
\subsection*{Bot}
Un bot è un software progettato per automatizzare attività ripetitive o interagire con gli utenti. 
Funziona seguendo regole predefinite o usando algoritmi di intelligenza artificiale.

\hypertarget{sec:branch}{}
\subsection*{Branch}
Letteralmente "ramo", indica un’entità che si sviluppa o si dirama da un punto principale. Nel contesto di un sistema di controllo delle versioni, 
un branch rappresenta una linea di sviluppo separata. Può essere utilizzato per sviluppare nuove funzionalità, risolvere bug o implementare modifiche 
senza influenzare direttamente il ramo principale del codice, noto come master o main.

\hypertarget{sec:browser}{}
\subsection*{Browser}
Applicazione software progettata per consentire agli utenti di navigare in Internet, visualizzare pagine web e accedere a contenuti online.

\hypertarget{sec:build}{Build}
\subsection*{Build}
Definito anche come costruzione, è il processo di compilazione di un progetto software, in cui il codice sorgente viene trasformato in un 
formato eseguibile. La build può includere una serie di attività come la compilazione del codice sorgente, la creazione di file di 
configurazione e la generazione di file di installazione o di pacchetti per la distribuzione del software. Normalmente viene automatizzato 
tramite l’utilizzo di strumenti affidabili e riproducibili, gestendo le dipendenze del progetto e automatizzando il processo di rilascio e 
il controllo di versione.

\hypertarget{sec:diagramma_di_burndown}{Burndown (diagramma di)}
\subsection*{Burndown, diagramma di}
Strumento grafico utilizzato nella gestione agile dei progetti per tracciare la quantità di lavoro rimanente nel tempo. 
Esso mostra la diminuzione progressiva (“burn down”) delle attività o dei punti stima rimanenti nel corso del tempo, 
consentendo al team di progetto di valutare il proprio progresso e adattare la pianificazione in base alle esigenze. 
A differenza del diagramma di Gantt, il diagramma di Burndown si concentra sulla visualizzazione dell’avanzamento reale rispetto al piano temporale.

\newpage


% Intestazione

\fancyhead[L]{C} % Testo a sinistra

\section{}

\hypertarget{sec:Camel Case}{}
\subsection*{Camel Case}
È una convenzione di scrittura usata nella programmazione per nominare variabili, funzioni e altri identificatori. 
Si distingue perché la prima parola inizia con una lettera minuscola, mentre le parole successive iniziano con una lettera maiuscola, senza spazi o separatori.

\hypertarget{sec:capitolato}{}
\subsection*{Capitolato}
Documento privato tra chi commissiona il lavoro e il gruppo (ditta) che lo esegue, in cui viene esposto un problema che il proponente necessita di risolvere 
e specifica le norme e i vincoli da rispettare per lo sviluppo del prodotto software specifico.

\hypertarget{sec:caso_uso}{}
\subsection*{Caso d'uso}
Descrizione dettagliata di come un utente (attore) interagisce con l'applicazione per il compimento di un'attività specifica. È uno strumento utilizzato nel 
contesto dello sviluppo software per individuare i requisiti funzionali del prodotto e per fornire una visuale chiara delle interazioni che possono avvenire 
all'interno dell'applicazione

\hypertarget{sec:chatgpt}{}
\subsection*{ChatGPT}
ChatGPT è un modello di intelligenza artificiale sviluppato da OpenAI, basato sulla famiglia di modelli GPT (Generative Pre-trained Transformer). 
È progettato per comprendere e generare testo in linguaggio naturale, rendendolo utile in numerosi scenari, come chatbot, assistenti virtuali, generazione 
di contenuti e risposte automatizzate. ChatGPT utilizza una vasta base di conoscenza pre-addestrata e può essere ulteriormente personalizzato per 
applicazioni specifiche, offrendo interazioni conversazionali fluide e contestualmente rilevanti.

\subsection*{Checklist}
Lista dettagliata di elementi, attività o criteri specifici che devono essere controllati, esaminati o completati durante le diverse fasi del ciclo di vita 
del software. E' utilizzata come strumento di controllo e verifica.

\hypertarget{sec:chroma}{}
\subsection*{Chroma}
Chroma è un database open-source ottimizzato per la gestione di dati vettoriali, progettato per supportare applicazioni di intelligenza artificiale e 
machine learning. È utilizzato principalmente per il retrieval di informazioni basato su similarità, come la ricerca di embedding, e integra funzionalità 
avanzate per lavorare con modelli di linguaggio (LLM). Fornisce un'API semplice per memorizzare, indicizzare e interrogare dati multidimensionali, 
rendendolo adatto a scenari come motori di raccomandazione, sistemi di domande e risposte, o clustering.

\hypertarget{sec:ciclo_di_vita}{}
\subsection*{Ciclo di vita del software}
Serie di fasi attraverso le quali un software passa dal suo concepimento iniziale fino al suo ritiro o dismissione. È un concetto chiave nell'ingegneria 
del software e fornisce una struttura organizzativa per il processo di sviluppo del software.

\hypertarget{sec:codifica}{}
\subsection*{Codifica}
Fase del processo di sviluppo software in cui gli sviluppatori traducono i requisiti e il design del sistema in linguaggio di programmazione, creando il 
codice sorgente. Durante questa fase, gli sviluppatori seguono gli standard di codifica e le buone prassi per assicurare la leggibilità, l’efficienza e la 
manutenibilità del codice.

\hypertarget{sec:confluence}{}
\subsection*{Confluence}
Piattaforma di collaborazione e gestione della conoscenza, utilizzata per creare, condividere e collaborare su documenti, progetti e informazioni all'interno 
di un team o di un'organizzazione. Confluence offre funzionalità come la creazione di wiki aziendali, la gestione di progetti, la documentazione tecnica e la 
collaborazione in tempo reale.

\hypertarget{sec:consuntivo}{}
\subsection*{Consuntivo}
Bilancio dei risultati ottenuti a rendiconto di un certo periodo temporale di attività, in termini di tempo e risorse.

\hypertarget{sec:contesto_applicativo}{}
\subsection*{Contesto applicativo}
Ambito o scenario in cui un'applicazione software è progettata per essere utilizzata. Il contesto applicativo definisce le condizioni, le esigenze e le
caratteristiche specifiche dell'ambiente in cui l'applicazione deve operare, influenzando il design, le funzionalità e le prestazioni del software.

\hypertarget{sec:controllo_versione}{}
\subsection*{Controllo di versione}
Strumento che consente di gestire e tracciare le modifiche apportate al codice sorgente o ad altri file di progetto.

\hypertarget{sec:cron}{}
\subsection*{Cron}
Nel contesto informatico, cron è un demone (un processo che si esegue in background) presente in molti sistemi operativi Unix-like 
(come Linux e macOS) che permette di pianificare l'esecuzione di comandi a intervalli di tempo regolari o a date e orari specifici.

\hypertarget{sec:css}{}
\subsection*{CSS}
Acronimo di Cascading Style Sheets, linguaggio di stile utilizzato per definire l’aspetto e lo stile delle pagine web HTML. CSS 
permette di separare il contenuto della pagina dalla sua presentazione, consentendo un maggiore controllo sulla formattazione e lo 
stile degli elementi della pagina. Con questo possono essere definiti degli stili, i quali possono definire proprietà come il colore, 
il font, la dimensione, la posizione e l’animazione.

\newpage