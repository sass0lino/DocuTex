% Intestazione
\fancyhead[L]{G} % Testo a sinistra

\section{}

\subsection*{Gantt, diagramma di}
Vedi \bulhyperlink{sec:diagramma_Gantt}{Diagramma di Gantt}.

\hypertarget{sec:git}{}
\subsection*{Git}
Software per il controllo di versione distribuito utilizzabile tramite interfaccia a riga di comando.

\hypertarget{sec:git_flow}{}
\subsection*{Git Flow}
Git Flow è un modello di branching di Git che definisce un insieme di regole e procedure per gestire il ciclo di vita di un progetto 
software. Offre una struttura ben definita per la collaborazione tra sviluppatori e la gestione dei rilasci, garantendo un flusso di 
lavoro efficiente e organizzato.

\subsection*{GitHub}
Servizio di hosting per progetti software. Il sito è principalmente utilizzato da sviluppatori che caricano il codice sorgente di programmi in dei 
repository e lo rendono scaricabile e migliorabile da altri sviluppatori. Questi ultimi possono interagire con i proprietari dei repository tramite un 
sistema per inviare segnalazioni di bug o richieste di funzionalità (issue tracker), un sistema per copiare il software in una versione modificabile 
(fork), un sistema per proporre modifiche agli sviluppatori originali (pull request) e un sistema di discussione legato al codice del repository (commenti).

\subsection*{GitHub Pages}
GitHub Pages è un servizio di hosting gratuito offerto da GitHub che permette agli utenti di creare e pubblicare facilmente siti web statici direttamente 
dai loro repository GitHub. È utilizzato comunemente per creare siti di documentazione, pagine personali o di progetto, blog e portali di portfolio. 
GitHub Pages è particolarmente apprezzato perché permette di ospitare un sito senza costi e con aggiornamenti automatici ogni volta che il repository 
viene modificato.

\subsection*{GitHub Projects}
GitHub Projects è uno strumento di gestione dei progetti integrato in GitHub, ideato per aiutare sviluppatori e team a organizzare, pianificare e tracciare 
il lavoro sui progetti direttamente all'interno dell'ambiente GitHub. Si basa su un sistema flessibile di "project board" simile a Kanban, che offre un modo 
visuale per coordinare i task e monitorare l’avanzamento del lavoro. GitHub Projects è uno strumento ideale per team che lavorano su progetti di sviluppo 
software, in quanto permette di gestire l’intero processo di sviluppo all'interno di GitHub stesso. Grazie alla sua integrazione nativa con il codice, le 
issues e le pull requests, aiuta a mantenere sincronizzati i task e a ridurre il contesto di cambiamento per gli sviluppatori, migliorando il flusso di 
lavoro e facilitando la collaborazione su GitHub.

\subsection*{Glossario}
Elenco organizzato di termini tecnici, acronimi e definizioni utilizzati nel contesto del progetto. Questo documento fornisce una chiara comprensione dei 
concetti e dei linguaggi specifici impiegati nel progetto, aiutando a ridurre ambiguità e fraintendimenti tra i membri del team e gli stakeholder.

\hypertarget{sec:google_chrome}{}
\subsection*{Google Chrome}
Browser web che consente di accedere a Internet e interagire con siti e applicazioni online.


\hypertarget{sec:google_meet}{}
\subsection*{Google Meet}
Servizio di videoconferenza sviluppato da Google. È parte di Google Workspace e offre funzionalità di videochiamata, chat, condivisione d
ello schermo e registrazione delle riunioni. Google Meet è ampiamente utilizzato per incontri di lavoro, lezioni online, webinar e conferenze, 
grazie alla sua facilità d’uso, alla stabilità della connessione e alla possibilità di partecipare senza dover scaricare alcun software.

\subsection*{Gulpease, indice di}
Indice di leggibilità di un testo specificamente in lingua italiana, che utilizza il numero delle parole, delle frasi e delle lettere per facilitare il 
calcolo automatico della leggibilità.

\newpage


% Intestazione
\fancyhead[L]{H} % Testo a sinistra

\section{}

\hypertarget{sec:hosting}{}
\subsection*{Hosting}
Servizio che fornisce l’infrastruttura necessaria per rendere accessibili siti web, applicazioni o dati attraverso Internet.

\hypertarget{sec:html}{}
\subsection*{HTML}
Acronimo di HyperText Markup Language, linguaggio utilizzato per la creazione e l’impaginazione di documenti ipertestuali disponibili 
sul web. Consente di strutturare il contenuto delle pagine web utilizzando tag e attributi, permettendo la visualizzazione di testo, 
immagini, link e altri elementi multimediali nei browser web.

\newpage


% Intestazione
\fancyhead[L]{I} % Testo a sinistra

\section{}

\hypertarget{sec:ia}{}
\subsection*{IA}
Acronimo di intelligenza artificiale

\hypertarget{sec:ide}{}
\subsection*{IDE}
Acronimo di Integrated Development Environment, è un ambiente di sviluppo, ovvero un software che, in fase di programmazione, supporta i programmatori 
nello sviluppo e debugging del codice sorgente di un programma, segnalando errori di sintassi del codice direttamente in fase di scrittura, oltre a fornire 
una serie di strumenti e funzionalità di supporto alla fase stessa di sviluppo e debugging.

\hypertarget{sec:modello_incrementale}{Incrementale (modello di sviluppo)}
\subsection*{Incrementale, modello di sviluppo}
Approccio alla creazione di software che suddivide il progetto in piccoli segmenti o incrementi. 
Ogni incremento rappresenta una versione funzionante del software che include nuove funzionalità o miglioramenti rispetto alla versione precedente.

\hypertarget{sec:implementazione}{}
\subsection*{Implementazione}
L'implementazione è il processo mediante il quale un progetto, un piano o un sistema viene realizzato e reso operativo, traducendo 
specifiche o requisiti in soluzioni concrete e funzionanti. In termini generali, implica la costruzione, l'integrazione e la 
configurazione degli elementi necessari per ottenere il risultato desiderato. Oltre a ciò, l'implementazione include la responsabilità 
di restituire un utensile usabile ai fini attesi, ossia fornire un prodotto che soddisfi le esigenze per cui è stato progettato, 
assicurandosi che sia funzionale, accessibile e adatto all'uso previsto dagli utenti finali.

\hypertarget{sec:intelligenza_artificiale}{}
\subsection*{Intelligenza artificiale}
Disciplina che studia come realizzare sistemi informatici in grado di simulare il pensiero umano. Sistemi basati su di essa hanno dunque l'abilità di mostrare 
capacità umane quali il ragionamento, l'apprendimento, la pianificazione e la creatività.

\hypertarget{ISO/IEC 31000:2018}{}
\subsection*{ISO/IEC 31000:2018}
La norma ISO 31000 "Risk management - Principles and guidelines", 
in italiano UNI ISO 31000 Gestione del rischio - Principi e linee guida. 
È una guida che fornisce principi e linee guida generali per la gestione del rischio. 
Può essere utilizzata da qualsiasi organizzazione pubblica, privata o sociale, associazione, gruppo o individuo, e non è specifica per nessuna industria o settore.
La ISO 31000 può essere applicata nel corso dell'intero ciclo di vita di un'organizzazione, ed essere adottata per molte attività come la definizione di strategie e decisioni, operazioni, processi, funzioni, progetti, prodotti, servizi e beni.
Può inoltre essere applicata a qualsiasi tipo di rischio, sia per conseguenze di tipo positivo che negativo. 

\hypertarget{ISO/IEC 9126}{}
\subsection*{ISO/IEC 9126}
La norma ISO/IEC 9126 è uno standard internazionale per la valutazione della qualità del software.
Definisce un modello di qualità software che si basa su sei caratteristiche generali di qualità (funzionalità, affidabilità, usabilità, efficienza,
manutenibilità, portabilità) e su 27 sotto-caratteristiche.
Lo standard è stato sostituito dalla norma ISO/IEC 25010:2011, che definisce un modello di qualità software più ampio e aggiornato.

\hypertarget{ISO/IEC/IEEE 12207}{}
\subsection*{ISO/IEC/IEEE 12207}
Lo standard ISO 12207 stabilisce un processo di ciclo di vita del software, compreso processi ed attività relative alle specifiche ed alla configurazione di un sistema.
Ad ogni processo corrisponde un insieme di risultati (outcome): in totale ci sono 43 processi, 133 attività, 325 sottoattività e 236 risultati (la nuova ISO/IEC 12207:2008 definisce 43 sistemi e processi software).
Lo standard ha come obiettivo principale quello di fornire una struttura comune che permetta a clienti, fornitori, sviluppatori, tecnici, manager di usare gli stessi termini e lo stesso linguaggio per definire gli stessi processi.
La struttura dello standard è stata concepita per essere flessibile e modulare in modo che sia adattabile alle necessità di chiunque lo utilizzi.

\subsection*{Issue}
Una issue su GitHub (e altre piattaforme di gestione del codice e dei progetti) è un elemento utilizzato per tracciare problemi, richieste di funzionalità, 
idee o miglioramenti relativi a un progetto. È uno strumento fondamentale per organizzare il lavoro collaborativo e garantire che tutti i membri del team 
siano aggiornati sui task e le priorità. Ogni issue rappresenta un singolo elemento che richiede attenzione o azione, e fornisce un luogo centralizzato per 
discuterlo, seguirlo e risolverlo.

\hypertarget{sec:issue_tracking_system}{Issue Tracking System (ITS)}
\subsection*{Issue Tracking System (ITS)}
Un Issue Tracking System (ITS) è un software o un sistema di gestione che consente di creare, monitorare e gestire issue 
(problemi, task, richieste o bug) all'interno di un progetto. Questo strumento è utilizzato principalmente per facilitare la collaborazione 
tra membri del team e per garantire che tutte le problematiche e le richieste siano gestite in modo organizzato, trasparente e tracciabile. 
Un ITS è spesso impiegato da team di sviluppo software, ma può essere utile anche in contesti di supporto clienti, gestione di progetto, e 
qualsiasi ambito dove occorra tracciare richieste di lavoro o problemi.

\newpage