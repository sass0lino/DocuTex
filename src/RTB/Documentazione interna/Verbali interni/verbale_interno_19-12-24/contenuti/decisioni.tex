% Intestazione
\fancyhead[L]{4 \hspace{0.2cm} Decisioni} % Testo a sinistra

\section{Decisioni}

Durante la riunione sono state prese le seguenti decisioni:

\vspace{0.5cm}

\begin{table}[htbp]
    \centering
    \rowcolors{2}{lightgray}{white}
    \begin{tabular}{|c|p{0.8\textwidth}|}
        \hline
        \rowcolor[gray]{0.75}
        \textbf{Codice} & \textbf{Descrizione}\\
        \hline
        VI 20.1 & Si è deciso di utilizzare \emph{cron} per l'aggiornamento automatico dei dati, in particolare la libreria \emph{Python Crontab}.\\
        \hline
        VI 20.2 & Si è deciso di riscrivere o rimuovere i casi d'uso che sono stati segnalati come più vicini al "come" rispetto al "cosa", 
        e di ripensare gli attori di partenza di alcuni casi d'uso, trovando dei sottosistemi dedicati.\\
        \hline
        VI 20.3 & Si è deciso di sistemare i casi d'uso asincronamente seguendo la distribuzione già compiuta per la loro creazione, 
        e poi di svolgere un incontro sincrono a gruppi durante le vacanze natalizie.
        I due gruppi saranno composti da:
        \begin{itemize}
            \item Michael Fantinato, Federica Bolognini e Riccardo Stefani;
            \item Giacomo Loat e Davide Verzotto e Filippo Righetto.
        \end{itemize}\\
        \hline
        VI 20.4 & Si è deciso che d'ora in poi, se possibile, inserire sempre il riferimento all'id della issue di Jira corrispondente nelle commit
        di GitHub\\
        \hline
        VI 20.5 & Si è deciso di inserire nel \emph{PoC}\textsubscript{\textit{\textbf{G}}} solamente le funzionalità prioritarie e di
        media priorità stabiliteci da \emph{AzzurroDigitale} nell'incontro esterno, e di riservare le altre per il \emph{MVP}\textsubscript{\textit{\textbf{G}}}.\\
        \hline
        VI 20.6 & E' stata decisa la seguente distribuzione delle attività legate alla programmazione durante le vacanze natalizie:
        \begin{itemize}
            \item Riccardo Stefani si occuperà del \emph{Prompt Engineering} del \emph{chatbot};
            \item Giacomo Loat si occuperà dello studio della libreria \emph{Python Crontab} e di tutto ciò che riguarda l'aggiornamento automatico del database vettoriale;
            \item Michael si occuperà dell'implementazione di \emph{Flask} o \emph{FastAPI} per la creazione di un \emph{web server} in \emph{Angular} per la visualizzazione della chat.
        \end{itemize}\\
        \hline
        VI 20.7 & Si è deciso di affidare le attività di gestione del documento di \emph{Analisi dei Requisiti}\textsubscript{\textit{\textbf{G}}} durante le vacanze a 
        Federica Bolognini, Davide Verzotto e Filippo Righetto.\\
        \hline
        VI 20.8 & Si è deciso di mantenere l'organizzazione già avuta nella prima sprint per quanto riguarda la verifica dei documenti e il resto delle attività burocratiche.,
        in particolare:
        \begin{itemize}
            \item Riccardo Stefani dovrà creare i grafici riassuntivi dello sprint e Federica Bolognini dovrà inserirli nel \emph{Piano di Progetto};
            \item Michael Fantinato dovrà verificare il documento di \emph{Analisi dei Requisiti};
            \item Davide Verzotto dovrà verificare il documento di \emph{Piano di Qualifica}\textsubscript{\textit{\textbf{G}}};
            \item Giacomo Loat dovrà verificare il documento di \emph{Piano di Progetto}\textsubscript{\textit{\textbf{G}}};
            \item Filippo Righetto dovrà verificare il documento di \emph{Norme di Progetto}\textsubscript{\textit{\textbf{G}}};
            \item Riccardo Stefani dovrà creare la pull request del ramo develop verso il ramo main del \emph{repository} della Documentazione.
        \end{itemize}\\
        \hline
        VI 20.9 & Si è deciso di preventivare circa 10 ore a testa di lavoro durante le vacanze di Natale. \\
        \hline
        VI 20.10 & Si è deciso che Riccardo Stefani scriverà il verbale dell'incontro interno del 19/12/24. \\
        \hline
    \end{tabular}
\end{table}
