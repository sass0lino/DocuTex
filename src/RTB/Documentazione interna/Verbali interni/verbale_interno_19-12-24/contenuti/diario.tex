% Intestazione
\fancyhead[L]{3 \hspace{0.2cm} Diario della riunione} % Testo a sinistra

\section{Diario della riunione}

\begin{itemize}
    \item Racconto della riunione esterna appena conclusa con \emph{AzzurroDigitale} per aggiornamento di Michael Fantinato che non ha potuto
    partecipare;
    \item Ragionamento sulla questione dell'aggiornamento automatico basato sui consigli del 
    \emph{proponente}\textsubscript{\textit{\textbf{G}}}, cioè si è discusso sull'utilizzo di \emph{cron}\textsubscript{\textit{\textbf{G}}}, 
    in particolare sulla libreria \emph{Python Crontab}\textsubscript{\textit{\textbf{G}}}
    per il linguaggio \emph{Python}\textsubscript{\textit{\textbf{G}}}, e sulla visualizzazione del risultato dell'aggiornamento da
    parte dello sviluppatore e dell'utente;
    \item Ragionamento sulla possibilità di aiutare il chatbot a rispondere introducendo delle buone pratiche di scrittura e 
    gestione dei nostri \emph{repository}\textsubscript{\textit{\textbf{G}}}, ad esempio inserire l'id della 
    \emph{issue}\textsubscript{\textit{\textbf{G}}} di \emph{Jira}\textsubscript{\textit{\textbf{G}}} corrispondente nelle commit di \emph{GitHub}\textsubscript{\textit{\textbf{G}}};
    \item Ragionamento su quanto emerso con il professor Cardin nella riunione del 18/12/2024, in particolare sulla questione degli
    \emph{attori}\textsubscript{\textit{\textbf{G}}} che causano il \emph{trigger}\textsubscript{\textit{\textbf{G}}} dei casi d'uso che possono essere 
    \emph{sottosistemi}\textsubscript{\textit{\textbf{G}}} interni all'applicazione ma non sistemi esterni fuori dal nostro controllo, e sulla questione di alcuni casi d'uso
    che sono stati segnalati come più vicini al "come" rispetto al "cosa", e quindi vanno riscritti oppure rimossi;
    \item Distribuzione delle attività da svolgere durante le vacanze natalizie, in particolare le seguenti:
    \begin{itemize}
        \item \emph{Prompt Engineering}\textsubscript{\textit{\textbf{G}}} del \emph{chatbot}\textsubscript{\textit{\textbf{G}}}, 
        perchè risponda correttamente alle domande seguendo i requisiti del proponente, e tentativo di miglioramento della ricerca
        di \emph{similarità}\textsubscript{\textit{\textbf{G}}} tra domanda e documenti presenti nel \emph{database vettoriale}\textsubscript{\textit{\textbf{G}}};
        \item Studiare la libreria \emph{Python Crontab} e approfondire l'aggiornamento automatico del \emph{database vettoriale}\textsubscript{\textit{\textbf{G}}};
        \item Implementare \emph{Flask}\textsubscript{\textit{\textbf{G}}} oppure \emph{FastAPI}\textsubscript{\textit{\textbf{G}}} per la creazione di un 
        \emph{web server}\textsubscript{\textit{\textbf{G}}} in \emph{Angular}\textsubscript{\textit{\textbf{G}}} per la visualizzazione della chat.
    \end{itemize}
    \item Scrittura del preventivo per la terza sprint, in particolare si è deciso di fare una stima di circa 10 ore a testa durante
    le vacanze natalizie;
    \item Organizzazione delle ultime operazioni conclusive della seconda sprint, in particolare:
    \begin{itemize}
        \item Creare i grafici riassuntivi dello sprint e inserirli nel \emph{Piano di Progetto}\textsubscript{\textit{\textbf{G}}};
        \item Fare una \emph{verifica}\textsubscript{\textit{\textbf{G}}} dei documenti allo stato attuale per poter fare una \emph{pull request}\textsubscript{\textit{\textbf{G}}} del ramo develop verso il ramo main nel repo della documentazione.
    \end{itemize}
\end{itemize}
