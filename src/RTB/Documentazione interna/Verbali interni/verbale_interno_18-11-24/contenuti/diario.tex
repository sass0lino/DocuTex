% Intestazione
\fancyhead[L]{3 \hspace{0.2cm} Diario della riunione} % Testo a sinistra

\section{Diario della riunione}

\begin{itemize}
    \item È stata valutata l'opzione di iniziare ad utilizzare un ramo develop nella repository;
    \item Si è discusso sulle task nella sezione to-do e sulla possibilità di inserirne di prive di assegnatario;
    \item Si è discusso dei costi in denaro per l'accesso del chatbot a Confluence,Jira e Github, tenendo conto che dovrebbero sarebbero da inserire nell'analisi dei costi;
    \item Si è discussa l'utilità di Next.js;
    \item Chi ha ricercato sugli \emph{LLM}\textsubscript{\textit{\textbf{G}}} ha presentato quello che ha trovato. Si è quindi parlato dei vari modelli utilizzabili, dei vantaggi e degli svantaggi, soprattutto in termini di costi. Inoltre è stato fatto presente come \emph{OpenAI}\textsubscript{\textit{\textbf{G}}} sarebbe più conveniente in termini di difficoltà di apprendimento, essendoci un gran numero di tutorial accessibili. È stato notato come Next.js sia un framework full-stack (sia back-end sia front-end) molto utilizzato nei tutorial, e quindi valutabile;
    \item È stata mostrata e discussa la stima sul costo per token per OpenAI come modello di chatbot;
    \item Si è valutato l'attuale stato della documentazione;
    \item Sono stati mostrati a schermo i software di Confluence e Jira, per valutarne l'introduzione come strumenti per il progetto e discuterene alcuni ostacoli legati all'utilizzo futuro durante l'addestramenteo del chatbot.
    Per Jira sono state prese in considerazione le seguenti caratteristiche:
    \begin{itemize}
    \item la visualizzazione a timeline è utilissima per farci gli screen per il piano di progetto;
    \item è possibile collegarsi a Github, quindi si possono mettere l'ID nel commit nello stesso modo solito;
    \item si possono creare dashboard, che supportano sia \emph{Gantt Chart}\textsubscript{\textit{\textbf{G}}}t che \emph{Burndown Chart}\textsubscript{\textit{\textbf{G}}};
    \item si possono creare delle \emph{sprint}\textsubscript{\textit{\textbf{G}}} e si può gestire perfettamente il loro \emph{backlog}\textsubscript{\textit{\textbf{G}}}.
    \end{itemize}
    Ricordando che gli sprint sono utilizzati solo in un progetto di tipo "Scrum".
    Per Confluence sono state prese in considerazione le seguenti caratteristiche:
    \begin{itemize}
        \item non fa solo documenti. È dotato di altre funzionalità come ad esempio brainstorming.
        \item non si possono convertire PDF in "pagine confluence", per caricare un PDF bisogna seguire uno di questi 3  metodi:
        \begin{itemize}
            \item utilizzare un link intelligente verso un pdf caricato online;
            \item visualizzarlo come anteprima con opzioni di scorrimento quando si allega un file ad una pagina;
            \item creare una pagina ed inserire all'interno un link ad un pdf online;
        \end{itemize}
    \end{itemize}
    Per entrambi (sia Jira sia Confluence, ma in generale per tutti i prodotti Atlassian), bisogna fare attenzione al dominio che viene creato all'inizio e che è presente nella barra degli indirizzi, perché poi è da lì che si accede a quei portali.
    \item Sono stati valutati NestJS e Spring Boot, oltre che Next.js (già preso in considerazione).
    \item È stato valutato quale LLM utilizzare, a fronte delle stime fatte dei costi in confronto coi vantaggi/svantaggi dei vari linguaggi;
    \item Valutato il piano generale di lavoro e la stima dei costi.
    \item Discusso se iniziare a stendere i casi d'uso, in particolare viene proposto di suddividerci in macro gruppi per approfondirli;
    \item Valutate le pratiche della prima sprint:
    \begin{itemize}
        \item raccogliere i dati delle ore che pensiamo di impiegare in ogni ruolo;
        \item distribuire i ruoli, stabilendo il giro da svolgere;
    \end{itemize}
    \item Pianificate le prossime riunioni.
    \item Stese domande da fare ad \emph{AzzurroDigitale}\textsubscript{\textit{\textbf{G}}}:
    \begin{itemize}
        \item chiedere se c'è un motivo per cui non hanno consigliato \emph{Python}\textsubscript{\textit{\textbf{G}}};
        \item citare Next.js come \emph{framework}\textsubscript{\textit{\textbf{G}}} full-stack (sia \emph{back-end}\textsubscript{\textit{\textbf{G}}} sia \emph{front-end}\textsubscript{\textit{\textbf{G}}}): può essere un vantaggio?
        \item cosa intendevano con "è possibile caricare documenti PDF con Confluence"? 
        \item può essere un problema il fatto che per recuperare i PDF serve entrare all'interno di una "pagina" o di un link?
    \end{itemize}








\end{itemize}