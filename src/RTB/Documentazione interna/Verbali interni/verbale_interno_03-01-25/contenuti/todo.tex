% Intestazione
\fancyhead[L]{5 \hspace{0.2cm} Todo} % Testo a sinistra

\section{Todo}

Durante la riunione sono emersi i seguenti task da svolgere:

\vspace{0.5cm}

\begin{table}[htbp]
\centering
\rowcolors{2}{lightgray}{white}
\begin{tabular}{|c|c|p{0.25\textwidth}|p{0.4\textwidth}|}
    \hline
    \rowcolor[gray]{0.75}
    \textbf{Codice} & \textbf{Dalla decisione} & \textbf{Assegnatario} & \textbf{Task Todo} \\
    \hline
    BUD-175 & VI 22.1 & Giacomo Loat & Esplorare le migliori soluzioni per creare l'immagine Docker per il backend, puntando
    all'ottimizzazione delle risorse\\
    \hline
    BUD-176 & VI 22.1 & Michael Fantinato & Esplorare le migliori soluzioni per creare l'immagine Docker per il frontend\\
    \hline
    BUD-177 & VI 22.2 & Riccardo Stefani & Provare a pesare la similarità basandosi sulla lunghezza dei documenti, in modo da dare 
    più importanza alla similarità con documenti lunghi rispetto a quella con documenti corti\\
    \hline
    BUD-178 & VI 22.3 & Tutto il gruppo & Studiare la tecnologia Docker\\
    \hline
    BUD-179 & VI 22.4 & Davide Verzotto & Creare la presentazione per il prossimo diario di bordo di mercoledì 08/01/25\\
    \hline
    BUD-180 & VI 22.8 & Riccardo Stefani & Fare gli screenshot del Diagramma di Gantt e del Diagramma di Burndown della terza sprint\\
    \hline
    BUD-181 & VI 22.8 & Federica Bolognini & Trascrivere il riassunto della terza sprint e il preventivo e consuntivo nel Piano di Progetto\\
    \hline
    BUD-182 & VI 22.9 & Riccardo Stefani & Scrivere il verbale della riunione interna del 03/01/25\\
    \hline
    BUD-183 & VI 22.9 & Riccardo Stefani & Creare la nuova Sprint su \emph{Jira}\textsubscript{\textit{\textbf{G}}}\\
    \hline
\end{tabular}
\end{table}