% Intestazione
\fancyhead[L]{3 \hspace{0.2cm} Diario della riunione} % Testo a sinistra

\section{Diario della riunione}

\begin{itemize}
    \item Abbiamo discusso su quanto emerso nell'incontro con il proponente \emph{AzzurroDigitale}, nel quale abbiamo deciso di
    utilizzare \emph{Docker}\textsubscript{\textit{\textbf{G}}} per la creazione di ambienti di sviluppo isolati e per favorire
    la collaborazione tra di noi, evitando di dover installare software aggiuntivo sulle nostre macchine. Inoltre, abbiamo discusso
    su come gestire le immagini Docker che ci serviranno, e ci siamo detti che è conveniente utilizzare una immagine base per il
    \emph{backend}\textsubscript{\textit{\textbf{G}}} e una per il \emph{frontend}\textsubscript{\textit{\textbf{G}}}, in modo da 
    poter lavorare con soluzioni ottimizzate per il servizio specifico;
    \item Abbiamo discusso su come gestire il problema della distanza di \emph{similarità}\textsubscript{\textit{\textbf{G}}}, che 
    è troppo alta per i documenti lunghi rispetto ai documenti corti. Abbiamo valutato alcune possibilità, e ci siamo detti che
    potremmo provare a pesare la similarità basandoci sulla lunghezza dei documenti, in modo da dare più importanza alla similarità
    con documenti lunghi rispetto a quella con documenti corti;
    \item Ci siamo detti che con il \emph{Piano di Progetto}\textsubscript{\textit{\textbf{G}}} e con il \emph{Glossario}\textsubscript{\textit{\textbf{G}}}
    siamo messi bene, con le \emph{Norme di Progetto}\textsubscript{\textit{\textbf{G}}} un po' meno  ma ci stiamo lavorando, mentre 
    con l'\emph{Analisi dei Requisiti}\textsubscript{\textit{\textbf{G}}} ed il \emph{Piano di Qualifica}\textsubscript{\textit{\textbf{G}}} 
    siamo in ritardo, e quindi ci impegneremo a terminare questi documenti il prima possibile;
    \item Ci siamo detti che per il prossimo diario di bordo di mercoledì 08/01/25 potremmo parlare di come gestire i rami
    del nostro \emph{repository}\textsubscript{\textit{\textbf{G}}}: ci siamo chiesti se è accettabile avere dei rami secondari instabili,
    oppure se anche i rami secondari devono sottostare ad un minimo controllo di \emph{qualità}\textsubscript{\textit{\textbf{G}}};
    \item Abbiamo compilato il consuntivo della terza sprint, utilizzando il \emph{Foglio Google}\textsubscript{\textit{\textbf{G}}} dedicato;
    \item Abbiamo compilato il preventivo per la quarta sprint, cercando di bilanciare il numero di ore tra i vari membri del gruppo,
    visto che si avvicina la sessione d'esami e quindi non tutti potranno dedicare lo stesso tempo al progetto;
    \item Abbiamo discusso sull'esecuzione delle procedure formali di conclusione della terza sprint, in particolare sulla verifica
    dei documenti, e ci siamo detti che al momento ci sono altre priorità, e quindi per questa volta faremo a meno di questa verifica. 
    Tuttavia, ci siamo detti, negli ultimi giorni della prossima sprint sarà necessario svolgere questa verifica in modo più
    approfondito del solito, per verificare anche il materiale della Sprint precedente e perchè saremo in vista della revisione
    \emph{RTB}\textsubscript{\textit{\textbf{G}}}. Invece, è giusto che proceda normalmente la trascrizione del
    riassunto della sprint e del preventivo e consuntivo nel Piano di Progetto.
    \item Abbiamo discusso sull'incontro con il professor Cardin, e ci siamo detti che aspetteremo la sua risposta entro martedì
    07/01/25, poi gli reinvieremo una mail. Il prossimo incontro del gruppo \emph{SWEg Labs} avverrà successivamente all'incontro con il professor
    Cardin, quindi ancora non ne conosciamo esattamente la data, ma ci aggiorneremo via \emph{Telegram}\textsubscript{\textit{\textbf{G}}}.
 \end{itemize}
