% Intestazione
\fancyhead[L]{5 \hspace{0.2cm} Todo} % Testo a sinistra

\section{Todo}

Durante la riunione sono emersi i seguenti task da svolgere:

\vspace{0.5cm}

\begin{table}[htbp]
\centering
\rowcolors{2}{lightgray}{white}
\begin{tabular}{|c|c|p{0.25\textwidth}|p{0.4\textwidth}|}
    \hline
    \rowcolor[gray]{0.75}
    \textbf{Codice} & \textbf{Dalla decisione} & \textbf{Assegnatario} & \textbf{Task Todo} \\
    \hline
    BUD-85 & VI 18.3 & Filippo Righetto & Scrivere gli attori nel documento \emph{Analisi dei Requisiti}. \\
    \hline
    BUD-86 & VI 18.4 & Tutto il gruppo & Richiedere un incontro online con il professore Cardin. \\
    \hline
    BUD-87 & VI 18.5 & Federica Bolognini, Filippo Righetto e Davide Verzotto & Tradurre i \emph{casi d'uso} in \emph{requisiti}. \\
    \hline
    BUD-88 & VI 18.5 & Riccardo Stefani & Sistema il suo piccolo lavoro svolto, inserendo le docstring. \\
    \hline
    BUD-89 & VI 18.5 & Riccardo Stefani & Testare la comprensione del \emph{bot} e aggiungere un header nel \emph{prompt}. \\
    \hline
    BUD-90 & VI 18.5 & Giacomo Loat & Aggiornamento automatico partendo dall'app. \\
    \hline
    BUD-91 & VI 18.8 & Michael Fantinato, Riccardo Stefani e Giacomo Loat & Pianifare delle strategie di programmazione.\\
    \hline
    BUD-92 & VI 18.9 & Riccardo Stefani & Creazione della repository per il \emph{PoC}.\\
    \hline
    BUD-93 & VI 18.10 & Tutto il gruppo & Studio di \emph{Angular} in parallelo con il \emph{PoC}. \\
    \hline
    BUD-94 & VI 18.12 & Tutto il gruppo  & Stabilire un avanzamento minimo per il periodo natalizio.\\
    \hline
    BUD-95 & VI 18.13 & Federica Bolognini  & Redigere verbale interno del 11-12-24\\
\end{tabular}
\end{table}