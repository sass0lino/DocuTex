% Intestazione
\fancyhead[L]{5 \hspace{0.2cm} Todo} % Testo a sinistra


\section{Todo}

Durante la riunione sono emersi i seguenti task da svolgere:

\vspace{0.5cm}

\begin{table}[htbp]
\centering
\rowcolors{2}{lightgray}{white}
\begin{tabular}{|c|c|p{0.25\textwidth}|p{0.4\textwidth}|}
    \hline
    \rowcolor[gray]{0.75}
    \multicolumn{1}{|c|}{\textbf{Codice}} & \multicolumn{1}{|c|}{\textbf{Dalla decisione}} & \multicolumn{1}{|c|}{\textbf{Assegnatario}} 
    & \multicolumn{1}{|c|}{\textbf{Task Todo}} \\
    \hline
    \#12 & VI 7.1 & Riccardo Stefani & Redigere verbale riunione \emph{07/11/24} \\
    \hline
    \#13 & VI 8.1 & Davide Verzotto & Scrivere l'introduzione e le parti descrittive dei processi delle \emph{Norme di Progetto} \\
    \hline
    \#14 & VI 8.1 & Riccardo Stefani & Scrivere la parte della documentazione nella sezione dei processi di supporto delle \emph{Norme di Progetto} \\
    \hline
    \#15 & VI 8.1 & Filippo Righetto & Scrivere l'introduzione e la descrizione generale dell'\emph{Analisi dei Requisiti} \\
    \hline
    \#16 & VI 8.1 & Giacomo Loat & Scrivere le definizioni di alcuni termini che è molto probabile che verranno inseriti nel \emph{Glossario} \\
    \hline
    \#17 & VI 8.1 & Federica Bolognini & Scrivere l'introduzione e iniziare l'analisi dei rischi del \emph{Piano di Progetto} \\
    \hline
    \#18 & VI 8.1, VI 8.2 & Michael Fantinanto & Scrivere l'introduzione, gli obiettivi metrici di qualità della documentazione e tutta
    la \emph{checklist}\textsubscript{\textit{\textbf{G}}} del \emph{Piano di Qualifica} \\
    \hline
    \#19 & VI 8.4 & Riccardo Stefani & Informarsi su come creare i grafici e i diagrammi di progetto utilizzando \emph{Fogli Google} \\
    \hline
    \#20 & VI 8.5 & Riccardo Stefani & Creare le \emph{issues}\textsubscript{\textit{\textbf{G}}} facenti seguito alle decisioni prese in riunione, e 
    configurare adeguatamente il GitHub Project \emph{BuddyBot} perchè in esso si visualizzi correttamente il 
    \emph{Gantt Chart}\textsubscript{\textit{\textbf{G}}} \\
    \hline
\end{tabular}
\end{table}
