% Intestazione
\fancyhead[L]{3 \hspace{0.2cm} Diario della riunione} % Testo a sinistra

\section{Diario della riunione}

\begin{itemize}
    \item La riunione è stata registrata per testare il software \emph{Whisper.AI}, che sarà usato per generare un resoconto della riunione,
    la cui qualità verrà valutata alla riunione successiva, e si deciderà allora se sfruttare questa applciazione in modo stabile.
    \item Utilizzando i verbali sono state spiegate a Federica Bolognini le attività svolte nei giorni della sua assenza.
    \item Sono state valutate le alternative a disposizione per il linguaggio di \emph{back-end}, sfruttando test pratici fatti in settimana e altra documentazione.
    \item Dopo una discussione sulle alternative a disposizione per il framework da usare per gestire il \emph{front-end}, si è deciso di aspetttare qualche giorno e
    rinviare la decisione all'incontro di lunedì prossimo.
    \item Sono state viste alcune alternative per la scelta delle tecnologie per la gestione del \emph{database vettoriale}, 
    si è deciso di approfondirle per prendere una decisione definitiva entro la fine della prossima settimana.
    \item È stato stabilito che si può proseguire con la stesura  dell’\emph{Analisi dei Requisiti}\textsubscript{\textit{\textbf{G}}} e del \emph{Piano di Qualifica}\textsubscript{\textit{\textbf{G}}}. 
    Inoltre, si è detto che è possibile trascrivere il preventivo orario della prima sprint nel 
    \emph{Piano di Progetto}\textsubscript{\textit{\textbf{G}}}.
    \item Sono state stabilite le sezioni dei documenti che si possono scrivere e i \emph{task} per la loro redazione sono stati distribuiti tra i membri del gruppo.
    \item È stata creata la presentazione per il \emph{Diario di Bordo} di lunedì \emph{2 dicembre} ed è stata scelta Federica Bolognini come relatrice.
    \item È stato proposto e accettato di stabilire delle date fisse per gli incontri interni, invece di stabilire di volta in volta data e 
    ora dell’incontro successivo. Sono stati scelti il lunedì pomeriggio e il giovedì (subito dopo l’incontro periodico con \emph{azzurroDigitale}) 
    come giorni fissi, mantenendo la possibilità di organizzare altre riunioni interne al bisogno.
\end{itemize}
